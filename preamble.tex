\usepackage[margin=1in]{geometry}
\usepackage{csquotes}
\usepackage{fancyhdr}
\usepackage{xr}
\usepackage{marginnote}
\usepackage{scrextend}
\usepackage[bottom]{footmisc}
\usepackage[style=apa]{biblatex}
\usepackage{enumitem}
\usepackage{siunitx}
\usepackage{subcaption,float}
\usepackage{tikz}
\usepackage{amsmath,amssymb,amsbsy,amsthm}
\usepackage{bm,physics,mathtools,nicematrix,empheq}
\usepackage[hidelinks]{hyperref}

\MakeOuterQuote{"}

\fancypagestyle{main}{
    \fancyhf{}
    \fancyhead[L]{\leftmark}
    \fancyhead[R]{MATH 20510}
    \fancyfoot[R]{Labalme\ \thepage}
}
\fancypagestyle{plain}{
    \fancyhead{}
    \renewcommand{\headrulewidth}{0pt}
}

\externaldocument{notes}

\reversemarginpar

\deffootnotemark{\textsuperscript{\textup{[}\thefootnotemark\textup{]}}}
\deffootnote[2.1em]{0em}{0em}{\textsuperscript{\thefootnote}}

\addbibresource{../main.bib}
\DefineBibliographyStrings{english}{bibliography={References}}

\setitemize[3]{label={\scriptsize$\blacksquare$}}
\setitemize[4]{label={\tikz[scale=0.06,baseline={(0,-0.14)}]{
    \draw [line width=0.3pt] (0,1) -- (1.2,0) -- (0,-1) -- (3.5,0) -- cycle;
    \fill (1.2,0) -- (0,-1) -- (3.5,0);
}}}

\usetikzlibrary{intersections,3d,bending,calc,decorations.markings}
\colorlet{brx}{brown}
\colorlet{gax}{gray}
\colorlet{rex}{red!80!black!90!orange!80}
\colorlet{blx}{blue!90!green!80}
\colorlet{orx}{orange!80!black!90!yellow!80}
\colorlet{grx}{green!50!black!90!yellow!80}

\DeclareMathOperator{\Hom}{Hom}
\DeclareMathOperator{\eva}{ev}
\DeclareMathOperator{\im}{im}
\DeclareMathOperator{\sgn}{sign}
\DeclareMathOperator{\Alt}{Alt}
\DeclareMathOperator{\spn}{span}
\DeclareMathOperator{\id}{id}
\DeclareMathOperator{\crl}{curl}
\DeclareMathOperator{\dvv}{div}
\DeclareMathOperator{\grd}{grad}
\DeclareMathOperator{\supp}{supp}
\DeclareMathOperator{\intt}{int}
\DeclareMathOperator{\Vol}{Vol}
\DeclareMathOperator{\vol}{vol}

\newcommand{\N}{\mathbb{N}}
\newcommand{\Z}{\mathbb{Z}}
\newcommand{\R}{\mathbb{R}}
\newcommand{\C}{\mathbb{C}}

\newcommand{\lin}[2][]{\mathcal{L}^{#1}(#2)}
\newcommand{\alt}[2][]{\mathcal{A}^{#1}(#2)}
\newcommand{\sym}[2][]{\mathcal{S}^{#1}(#2)}
\newcommand{\lam}[2][]{\Lambda^{#1}(#2)}
\newcommand{\ide}[2][]{\mathcal{I}^{#1}(#2)}
\newcommand{\ome}[2][]{\Omega^{#1}(#2)}

\newcommand{\e}[1][]{\text{e}^{#1}}

\usepackage{subfiles}