\documentclass[../notes.tex]{subfiles}

\pagestyle{main}
\renewcommand{\chaptermark}[1]{\markboth{\chaptername\ \thechapter\ (#1)}{}}
\setcounter{chapter}{3}

\begin{document}




\chapter{Differential Forms}
\section{Overview of Differential Forms}
\begin{itemize}
    \item \marginnote{4/18:}Office Hours on Wednesday, 4:00-5:00 PM.
    \item Plan:
    \begin{itemize}
        \item An impressionistic overview of what (differential) forms do/are.
        \item Tangent spaces.
        \item Vector fields/integral curves.
        \item 1-forms; a warm-up to $k$-forms.
    \end{itemize}
    \item Impressionistic overview of the rest of \textcite{bib:DifferentialForms}.
    \begin{itemize}
        \item An open subset $U\subset\R^n$; $n=2$ and $n=3$ are nice.
        \item Sometimes, we'll have some functions $F:U\to V$; this is where pullbacks come into play.
        \item At every point $p\in U$, we'll define a vector space (the tangent space $T_p\R^n$). Associated to that vector space you get our whole slew of associated spaces (the dual space $T_p^*\R^n$, and all of the higher exterior powers $\lam[k]{T_p^*\R^n}$).
        \item We let $\omega\in\ome[k]{U}$ be a $k$-form in the space of $k$-forms.
        \item $\omega$ assigns (smoothly) to every point $p\in U$ an element of $\lam[k]{T_p^*\R^n}$.
        \item Question: What really is a $k$-form?
        \begin{itemize}
            \item Answer: Something that can be integrated on $k$-dimensional subsets.
            \item If $k=1$, i.e., $\omega\in\ome[1]{U}$, then $U$ can be integrated over curves.
        \end{itemize}
        \item If we take $k=0$, then $\ome[0]{U}=C^\infty(U)$, i.e., the set of all smooth functions $f:U\to\R$.
        \begin{itemize}
            \item \textcite{bib:DifferentialForms} doesn't, but Klug will and we should distinguish between functions $F:U\to V$ and $f:U\to\R$.
        \end{itemize}
        \item We will soon construct a map $d:\ome[0]{U}\to\ome[1]{U}$ (the \textbf{exterior derivative}) that is rather like the gradient but not quite.
        \begin{itemize}
            \item $d$ is linear.
            \item Maps from vector spaces are heretofore assumed to be linear unless stated otherwise.
        \end{itemize}
        \item The 1-forms in $\im(d)$ are special: $\int_\gamma\dd{f}=f(\gamma(b))-f(\gamma(a))$ only depends on the endpoints of $\gamma:[a,b]\to U$! The integral is \emph{path-independent}.
        \item A generalization of this fact is that instead of integrating along the surface $M$, we can integrate along the boundary curve:
        \begin{equation*}
            \int_M\dd{\omega} = \int_{\partial M}\omega
        \end{equation*}
        This is \textbf{Stokes' theorem}.
        \begin{itemize}
            \item $M$ is a $k$-dimensional subset of $U\subset\R^n$.
        \end{itemize}
        \item Note that we have all manner of functions $d$ that we could differentiate between (because they are functions) but nobody does.
        \begin{equation*}
            0 \rightarrow \ome[0]{U}
            \xrightarrow{d} \ome[1]{U}
            \xrightarrow{d} \ome[2]{U}
            \xrightarrow{d} \cdots
            \xrightarrow{d} \ome[n]{U}
            \xrightarrow{d} 0
        \end{equation*}
        \item Theorem: $d^2=d\circ d=0$.
        \begin{itemize}
            \item Corollary: $\im(d^{n-1})\subset\ker(d^n)$.
        \end{itemize}
        \item We'll define $H^k_{dR}(U)=\ker(d)/\im(d)$.
        \begin{itemize}
            \item These will be finite dimensional, even though all the individual vector spaces will be infinite dimensional.
            \item These will tell us about the shape of $U$; basically, if all of these equal zero, $U$ is simply connected. If some are nonzero, $U$ has some holes.
        \end{itemize}
        \item For small values of $n$ and $k$, this $d$ will have some nice geometric interpretations (div, grad, curl, n'at).
        \item We'll have additional operations on forms such as the wedge product.
    \end{itemize}
    \item \textbf{Tangent space} (of $p$): The following set. \emph{Denoted by} $\bm{T_p\pmb{\R}^n}$. \emph{Given by}
    \begin{equation*}
        T_p\R^n = \{(p,v):v\in\R^n\}
    \end{equation*}
    \begin{itemize}
        \item This is naturally a vector space with addition and scalar multiplication defined as follows.
        \begin{align*}
            (p,v_1)+(p,v_2) &= (p,v_1+v_2)&
            \lambda(p,v) &= (p,\lambda v)
        \end{align*}
        \item The point is that
        \begin{equation*}
            T_p\R^n \neq T_q\R^n
        \end{equation*}
        for $p\neq q$ even though the spaces are isomorphic.
        \item Aside: $F:U\to V$ differentiable and $p\in U$ induce a map $\dd{F_p}:T_p\R^n\to T_{F(p)}\R^m$ called the "derivative at $p$."
        \begin{itemize}
            \item We will see that the matrix of this map is the Jacobian.
        \end{itemize}
        \item Chain rule: If $U\xrightarrow{F}V\xrightarrow{G}W$, then
        \begin{equation*}
            \dd{(G\circ F)_p} = \dd{G}_{F(p)}\circ\dd{F_p}
        \end{equation*}
    \end{itemize}
    \item This is round 1 of our discussion on tangent spaces.
    \item Round 2, later on, will be submanifolds such as $T_pM$: The tangent space to a point $p$ of a manifold $M$.
    \item \textbf{Vector field} (on $U$): A function that assigns to each $p\in U$ an element of $T_p\R^n$.
    \begin{itemize}
        \item A constant vector field would be $p\mapsto(p,v)$, visualized as a field of vectors at every $p$ all pointing the same direction. For example, we could take $v=(1,1)$.
        \emph{picture}
        \item Special case: $v=e_1,e_2,\dots,e_n$. Here we use the notation $e_i=\dv*{x_i}$.
        \item Example: $n=2$, $U=\R^2\setminus\{(0,0)\}$. We could take a vector field that spins us around in circles.
        \item Notice that for all $p$, $\dv*{x_1}|_p,\dots,\dv*{x_n}|_p\in T_p\R^n$ are a basis.
        \begin{itemize}
            \item Thus, any vector field $v$ on $U$ can be written uniquely as
            \begin{equation*}
                v = f_1\dv{x_1}+\cdots+f_n\dv{x_n}
            \end{equation*}
            where the $f_1,\dots,f_n$ are functions $f_i:U\to\R$.
        \end{itemize}
    \end{itemize}
\end{itemize}



\section{The Lie Derivative and 1-Forms}
\begin{itemize}
    \item \marginnote{4/20:}Plan:
    \begin{itemize}
        \item Vector fields and their integral curves.
        \item Lie derivatives.
        \item 1-forms and $k$-forms.
        \item $\ome[0]{U}\xrightarrow{d}\ome[1]{U}$.
    \end{itemize}
    \item Notation.
    \begin{itemize}
        \item $U\subset\R^n$.
        \item $v$ denotes a vector field on $U$.
        \begin{itemize}
            \item Note that the set of all vector fields on $U$ constitute the vector space ??.
        \end{itemize}
        \item $v_p\in T_p\R^n$.
        \item $\omega_p\in\lam[k]{T_p^*\R^n}$.
        \item $\dv*{x_i}|_p=(p,e_i)\in T_p\R^n$.
    \end{itemize}
    \item Recall that any vector field $v$ on $U$ can be written uniquely as
    \begin{equation*}
        v = g_1\dv{x_1}+\cdots+g_n\dv{x_n}
    \end{equation*}
    where the $g_i:U\to\R$.
    \item \textbf{Smooth} (vector field): A vector field $v$ for which all $g_i$ are smooth.
    \item From now on, we assume unless stated otherwise that all vector fields are smooth.
    \item \textbf{Lie derivative} (of $f$ wrt. $v$): The function $L_vf:U\to\R$ defined by $p\mapsto D_{v_p}(f)(p)$, where $v$ is a vector field on $U$ and $f:U\to\R$ (always smooth).
    \begin{itemize}
        \item Recall that $D_{v_p}(f)(p)$ denotes the directional derivative of $f$ in the direction $v_p$ at $p$.
        \item As some examples, we have
        \begin{align*}
            L_{\dv*{x_i}}f &= \dv{f}{x_i}&
            L_{(g_1\dv{x_1}+\cdots+g_n\dv{x_n})}f &= g_1\dv{f}{x_1}+\cdots+g_n\dv{f}{x_n}
        \end{align*}
    \end{itemize}
    \item Property.
    \begin{enumerate}
        \item Product rule: $L_v(f_1f_2)=(L_vf_1)f_2+f_1(L_vf_2)$.
    \end{enumerate}
    \item Later: Geometric meaning to the expression $L_vf=0$.
    \begin{itemize}
        \item Satisfied iff $f$ is constant on the integral curves of $v$. As if $f$ "flows along" the vector field.
    \end{itemize}
    \item We define $T_p^*\R^n=(T_p\R^n)^*$.
    \item 1-forms:
    \begin{itemize}
        \item A (differential) 1-form on $U\subset\R^n$ is a function $\omega:p\mapsto\omega_p\in T_p^*\R^n$.
        \item A "co-vector field"
    \end{itemize}
    \item Notation: $\dd{x_i}$ is the 1-form that at $p$ is $(p,e_i^*)\in T_p^*\R^n$.
    \item For example, if $U=\R^2$ and $\omega=\dd{x_1}$, then we have the vector field of "unit vectors pointing to the right at each point."
    \item Note: Given any 1-form $\omega$ on $U$, we can write $\omega$ uniquely as
    \begin{equation*}
        \omega = g_1\dd{x_1}+\cdots+g_n\dd{x_n}
    \end{equation*}
    for some set of smooth $g_i:U\to\R$.
    \item Notation:
    \begin{itemize}
        \item $\ome[1]{U}$ is the set of all smooth 1-forms.
        \item Notice that $\ome[1]{U}$ is a vector space.
    \end{itemize}
    \item Given $\omega\in\ome[1]{U}$ and a vector field $v$ on $U$, we can define $\omega(v):U\to\R$ by $p\mapsto\omega_p(v_p)$.
    \item If $U=\R^2$, we have that
    \begin{align*}
        \dd{x}\left( \dv{x} \right) &= 1&
        \dd{x}\left( \dv{y} \right) &= 0
    \end{align*}
    \item Note that $\dd{x},\dd{y}$ are not a basis for $\ome[1]{U}$ since the latter is infinite dimensional.
    \item Exterior derivative for 0/1 forms.
    \begin{itemize}
        \item Let $d:\ome[0]{U}\to\ome[1]{U}$ take $f:U\to\R$ to $\pdv{f}{x_1}\dd{x_1}+\cdots+\pdv{f}{x_n}\dd{x_n}$.
        \item This represents the gradient as a 1-form.
    \end{itemize}
    \item Check:
    \begin{enumerate}
        \item Linear.
        \item $\dd{x_i}=\dd{(x_i)}$, where $x_i:\R^n\to\R$ is the $i^\text{th}$ coordinate function.
    \end{enumerate}
\end{itemize}



\section{Integral Curves}
\begin{itemize}
    \item \marginnote{4/22:}Plan:
    \begin{itemize}
        \item Clear up a bit of notational confusion.
        \item Discuss integral curves of vectors fields.
        \item $k$-forms.
        \item Exterior derivatives $d:\ome[k]{U}\to\ome[k+1]{U}$ (definition and properties).
    \end{itemize}
    \item Notation:
    \begin{itemize}
        \item $F:\R^n\to\R^m$ smooth.
        \item We are used to denoting derivatives by big $D$: $DF_p:T_p\R^n\to T_{f(p)}\R^m$ where bases of the two spaces are $e_1,\dots,e_n$ and $e_1,\dots,e_m$ has matrix equal to the Jacobian:
        \begin{equation*}
            [DF_p] =
            \begin{bmatrix}
                {\dv{F_i}{x_j}}(p)
            \end{bmatrix}
        \end{equation*}
        \item The book often uses small $d$: $f:U\to\R$ has $df_p:T_p\R^n\to T_{f(p)}\R$, where the latter set is isomorphic to $\R$.
        \item $df:p\to df_p\in T_p^*\R^n$.
        \item Klug said
        \begin{equation*}
            \dd{f} = \sum_{i=1}^n\pdv{f}{x_i}\dd{x_i}
        \end{equation*}
        \item Homework 1 defined $\dd{f}=df$?
        \item Sometimes three perspectives help you keep this all straight:
        \begin{enumerate}
            \item Abstract nonsense: The definition of the derivative.
            \item How do I compute it: Apply the formula.
            \item What is it: E.g., magnitude of the directional derivative in the direction of steepest ascent.
        \end{enumerate}
    \end{itemize}
    \item For the homework,
    \begin{itemize}
        \item Let $\omega$ be a 1-form in $\ome[1]{U}$.
        \item Let $\gamma:[a,b]\to U$ be a curve in $U$.
        \item Then $\dd{\gamma_p}=\gamma_p':T_p\R\to T_{\gamma(p)}\R^n$ is a function that takes in points of the curve and spits out tangent vectors.
        \item Integrating swallows 1-forms and spits out numbers.
        \begin{equation*}
            \int_\gamma\omega = \int_a^b\omega(\gamma'(t))\dd{t}
        \end{equation*}
        \item Problem: If $\omega=\dd{f}$, then
        \begin{equation*}
            \int_\gamma\omega = f(\gamma(b))-f(\gamma(a))
        \end{equation*}
        regardless of the path.
        \item Question: Given a 1-form $\omega$, is $\omega=\dd{f}$ for some $f$?
        \item Homework: Explicit $U$, $\omega$, closed $\gamma$ such that $\int_\gamma\omega\neq 0$ implies that $\omega\neq\dd{f}$. This motivates and leads into the de Rham cohomology.
    \end{itemize}
    \item Aside: It won't hurt (for now) to think of 1-forms as vector fields.
    \item Integral curves: Let $U\subset\R^n$, $v$ be a (smooth) vector field on $U$. A curve $\gamma:(a,b)\to U$ is an \textbf{integral curve} for $v$ if $\gamma'(t)=v_{\gamma(t)}$.
    \item Examples:
    \begin{itemize}
        \item If $U=\R^2$ and $\gamma=\dv*{x}$, then the integral curve is the line from left to right traveling at unit speed. The curve has to always have as it's tangent vector the unit vector pointing right (which is the vector at every point in the vector field).
        \item Vector fields flow everything around. An integral curve is the trajectory of a particle subjected to the vector field as a force field.
    \end{itemize}
    \item Main points:
    \begin{enumerate}
        \item These integral curves always exist (locally) and often exist globally (cases in which they do are called \textbf{complete vector fields}).
        \item They are unique given a starting point $p\in U$.
    \end{enumerate}
    \item An incomplete vector field is one such as the "all roads lead to Rome" vector field where everything always points inward. This is because integral curves cannot be defined for all "time" (real numbers, positive and negative).
    \item The proofs are in the book; they require an existence/uniqueness result for ODEs and the implicit function theorem.
    \item Aside: $f:U\to\R$, $v$ a vector field, implies that $L_vf=0$ means that $f$ is constant along all the integral curves of $v$. This also means that $f$ is integral for $v$.
    \item \textbf{Pullback} (of 1-forms): If $F:U\to V$, $d:\ome[0]{U}\to\ome[1]{U}$, and $d:\ome[0]{V}\to\ome[1]{V}$, then we get an induced map $F^*:\ome[0]{V}\to\ome[0]{U}$. If $f:V\to\R$, then $f\circ F$ is involved.
    \begin{itemize}
        \item We're basically saying that if we have $\Hom(A,X)$ (the set of all functions from $A$ to $X$) and $\Hom(B,X)$, then if we have $F:A\to B$, we get an induced map $F^*:\Hom(B,X)\to\Hom(A,X)$ that is precomposed with $F$.
    \end{itemize}
\end{itemize}



\section{Chapter 2: Differential Forms}
\emph{From \textcite{bib:DifferentialForms}.}
\begin{itemize}
    \item \marginnote{5/5:}Goals for this chapter.
    \begin{itemize}
        \item Generalize to $n$ dimensions the basic operations of 3D vector calculus (\textbf{divergence}, \textbf{gradient}, and \textbf{curl}).
        \begin{itemize}
            \item $\dvv$ and $\grd$ are pretty straightforward, but $\crl$ is more subtle.
        \end{itemize}
        \item Substitute \textbf{differential forms} for \textbf{vector fields} to discover to a natural generalization of the operations, in particular, where all three operations are special cases of \textbf{exterior differentiation}.
    \end{itemize}
    \item Introducing vector fields and their dual objects (\textbf{one-forms}).
    \item \textbf{Tangent space} (to $\R^n$ at $p$): The set of pairs $(p,v)$ for all $v\in\R^n$. \emph{Denoted by} $\bm{T_p\pmb{\R}^n}$. \emph{Given by}
    \begin{equation*}
        T_p\R^n = \{(p,v)\mid v\in\R^n\}
    \end{equation*}
    \item Operations on the tangent space.
    \begin{itemize}
        \item Directly, we identify $T_p\R^n\cong\R^n$ by $(p,v)\mapsto v$ to make $T_p\R^n$ a vector space.
        \item Explicitly, we define
        \begin{align*}
            (p,v_1)+(p,v_2) &= (p,v_1+v_2)&
            \lambda(p,v) &= (p,\lambda v)
        \end{align*}
        for all $v,v_1,v_2\in\R^n$ and $\lambda\in\R$.
    \end{itemize}
    \item \textbf{Derivative} (of $f$ at $p$): The linear map from $\R^n\to\R^m$ defined by the following $m\times n$ matrix, where $U\subset\R^n$ is open and $f:U\to\R^m$ is a $C^1$-mapping. \emph{Denoted by} $\bm{Df(p)}$. \emph{Given by}
    \begin{equation*}
        Df(p) =
        \begin{bmatrix}
            \displaystyle{\pdv{f_i}{x_j}}(p)
        \end{bmatrix}
    \end{equation*}
    \item $\bm{\mathbf{d}f_p}$: The linear map from $T_p\R^n\to T_q\R^m$ defined as follows, where $U\subset\R^n$ open, $f:U\to\R^m$ is a $C^1$-mapping, and $q=f(p)$. \emph{Given by}
    \begin{equation*}
        \dd f_p(p,v) = (q,Df(p)v)
    \end{equation*}
    \begin{itemize}
        \item \textcite{bib:DifferentialForms} also refer to this as the "base-pointed" version of the derivative of $f$ at $p$.
    \end{itemize}
    \item The chain rule for the base-pointed version, where $U\subset\R^n$ open, $f:U\to\R^n$ is a $C^1$-mapping, $\im(f)\subset V$ open, and $g:V\to\R^k$ is a $C^1$-mapping.
    \begin{equation*}
        \dd g_q\circ\dd f_p = \dd(f\circ g)_p
    \end{equation*}
    \item \textbf{Vector field} (on $\R^3$): A function which attaches to each point $p\in\R^3$ a base-pointed arrow $(p,v)\in T_p\R^3$.
    \begin{itemize}
        \item These vector fields are the typical subject of vector calculus.
    \end{itemize}
    \item \textbf{Vector field} (on $U$): A function which assigns to each point $p\in U$ a vector in $T_p\R^n$, where $U\subset\R^n$ is open. \emph{Denoted by} $\bm{v}$.
    \begin{itemize}
        \item We denote the value of $\bm{v}$ at $p$ by either $\bm{v}(p)$ or $\bm{v}_p$.
    \end{itemize}
    \item \textbf{Constant} (vector field): A vector field of the form $p\mapsto(p,v)$, where $v\in\R^n$ is fixed.
    \item $\bm{\partial/\partial x_i}$: The constant vector field having $v=e_i$.
    \item $\bm{f\pmb{v}}$: The vector field defined on $U$ as follows, where $f:U\to\R$. \emph{Given by}
    \begin{equation*}
        p \mapsto f(p)\bm{v}(p)
    \end{equation*}
    \begin{itemize}
        \item Note that we are invoking our definition of scalar multiplication on $T_p\R^n$ here.
    \end{itemize}
    \item \textbf{Sum} (of $\bm{v}_1,\bm{v}_2$): The vector field on $U$ defined as follows. \emph{Denoted by} $\bm{\pmb{v}_1+\pmb{v}_2}$. \emph{Given by}
    \begin{equation*}
        p \mapsto \bm{v}_1(p)+\bm{v}_2(p)
    \end{equation*}
    \begin{itemize}
        \item Note that we are invoking our definition of addition on $T_p\R^n$ here.
    \end{itemize}
    \item The list of vectors $(\pdv*{x_1})_p,\dots,(\pdv*{x_n})_p$ constitutes a basis of $T_p\R^n$.
    \begin{itemize}
        \item Recall that $(\pdv*{x_i})_p=(p,e_i)$.
        \item Thus, if $\bm{v}$ is a vector field on $U$, it has a unique decomposition
        \begin{equation*}
            \bm{v} = \sum_{i=1}^ng_i\pdv{x_i}
        \end{equation*}
        where each $g_i:U\to\R$.
    \end{itemize}
    \item $\bm{C^\infty}$ (vector field): A vector field such that $g_i\in C^\infty(U)$ for all $g_i$'s in its unique decomposition.
    \item \textbf{Lie derivative} (of $f$ with respect to $\bm{v}$): The function from $U\to\R$ defined as follows, where $U\subset\R^n$, $f:U\to\R$ is a $C^1$-mapping, and $\bm{v}(p)=(p,v)$. \emph{Denoted by} $\bm{L_{\pmb{v}}f}$. \emph{Given by}
    \begin{equation*}
        L_{\bm{v}}f(p) = Df(p)v
    \end{equation*}
    \begin{itemize}
        \item A more explicit formula for the Lie derivative is
        \begin{equation*}
            L_{\bm{v}}f = \sum_{i=1}^ng_i\pdv{f}{x_i}
        \end{equation*}
        \item The vector field decides the direction in which we take the derivative at each point. Instead of having to take a derivative everywhere in one direction at a time, we can now take a derivative in a different direction at every point!
    \end{itemize}
    \item Lemma 2.1.11: Let $U$ be an open subset of $\R^n$, $\bm{v}$ a vector field on $U$, and $f_1,f_2\in C^1(U)$. Then
    \begin{equation*}
        L_{\bm{v}}(f_1\cdot f_2) = L_{\bm{v}}(f_1)\cdot f_2+f_1\cdot L_{\bm{v}}(f_2)
    \end{equation*}
    \begin{proof}
        See Exercise 2.1.ii.
    \end{proof}
    \item \textbf{Cotangent space} (to $\R^n$ at $p$): The dual vector space to $T_p\R^n$. \emph{Denoted by} $\bm{T_p^*\pmb{\R}^n}$. \emph{Given by}
    \begin{equation*}
        T_p^*\R^n = (T_p\R^n)^*
    \end{equation*}
    \item \textbf{Cotangent vector} (to $\R^n$ at $p$): An element of $T_p^*\R^n$.
    \item \textbf{Differential one-form} (on $U$): A function which assigns to each point $p\in U$ a cotangent vector. \emph{Also known as} \textbf{one-form} (on $U$). \emph{Denoted by} $\bm{\omega}$. \emph{Given by}
    \begin{equation*}
        p \mapsto \omega_p
    \end{equation*}
    \item Note that by identifying $T_p\R\cong\R$, we have that $\dd f_p\in T_p^*\R^n$, assuming that $f:U\to\R$.
    \begin{itemize}
        \item Geometric example: Consider $f:\R^2\to\R$ such that $f\in C^1$. By the latter condition, we know that the graph of $f$ is a "smooth" surface in $\R^3$, i.e., one without any abrupt changes in derivative (consider the graph of the piecewise function defined by $-x^2$ for $x<0$ and $x^2$ for $x\geq 0$, for example). What $\dd f_p$ does is take a point $(p_1,p_2,q)$, where $q=f(p)$, on the surface and a vector $v$ with tail at $(p,q)$, and give us a number representing the magnitude of the instantaneous change of $f$ at $p$ in the direction $v$. Thus, $\dd f_p$ contains, in a sense, all of the information concerning the rate of change of $f$ at $p$.
    \end{itemize}
    \item $\bm{\mathbf{d}f}$: The one-form on $U$ defined as follows. \emph{Given by}
    \begin{equation*}
        p \mapsto \dd f_p
    \end{equation*}
    \begin{itemize}
        \item Continuing with the geometric example: What $\dd f$ does is take every point $p$ across the surface and return all of the information concerning the rate of change of $f$ at $p$ (packaged neatly by $\dd f_p$).
    \end{itemize}
    \item \textbf{Pointwise product} (of $\phi$ with $\omega$): The one-form on $U$ defined as follows, where $\phi:U\to\R$ and $\omega$ is a one-form. \emph{Denoted by} $\bm{\phi\omega}$. \emph{Given by}
    \begin{equation*}
        (\phi\omega)_p = \phi(p)\omega_p
    \end{equation*}
    \item \textbf{Pointwise sum} (of $\omega_1,\omega_2$): The one-form on $U$ defined as follows. \emph{Denoted by} $\bm{\omega_1+\omega_2}$. \emph{Given by}
    \begin{equation*}
        (\omega_1+\omega_2)_p = (\omega_1)_p+(\omega_2)_p
    \end{equation*}
    \item $\bm{x_i}$: The function from $U\to\R$ defined as follows. \emph{Given by}
    \begin{equation*}
        x_i(u_1,\dots,u_n) = u_i
    \end{equation*}
    \begin{itemize}
        \item $x_i$ is constantly increasing in the $x_i$-direction, and constant in every other direction.
    \end{itemize}
    \item $\bm{(\mathbf{d}x_i)_p}$: The linear map from $T_p\R^n\to\R$ (i.e., the cotangent vector in $T_p^*\R^n$) defined as follows. \emph{Given by}
    \begin{equation*}
        (\dd x_i)_p(p,a_1x_1+\cdots+a_nx_n) = a_1
    \end{equation*}
    \begin{itemize}
        \item Naturally, the instantaneous change in $x_i$ at any point $p$ in the direction $\bm{v}(p)$ will just be the magnitude of $\bm{v}(p)$ in the $x_i$-direction.
        \item It follows immediately that
        \begin{equation*}
            (\dd x_i)_p\left( \pdv{x_j} \right)_p = \delta_{ij}
        \end{equation*}
    \end{itemize}
    \item Consequently, the list of cotangent vectors $(\dd x_1)_p,\dots,(\dd x_n)_p$ constitutes a basis of $T_p^*\R^n$ that is \textbf{dual} to the basis $(\pdv*{x_1})_p,\dots,(\pdv*{x_n})_p$ of $T_p\R^n$.
    \item $\bm{\mathbf{d}x_i}$: The one-form on $U$ defined as follows. \emph{Given by}
    \begin{equation*}
        p \mapsto (\dd x_i)_p
    \end{equation*}
    \begin{itemize}
        \item Thus, if $\omega_p:T_p\R^n\to\R$, it has a unique decomposition
        \begin{equation*}
            \omega_p = \sum_{i=1}^nf_i(p)(\dd x_i)_p
        \end{equation*}
        where every $f_i:U\to\R$.
        \item Similarly, $\omega:U\to T_p^*\R^n$ has a unique decomposition
        \begin{equation*}
            \omega = \sum_{i=1}^nf_i\dd x_i
        \end{equation*}
    \end{itemize}
    \item \textbf{Smooth} (one-form): A one form for which the associated functions $f_1,\dots,f_n\in C^\infty$. \emph{Also known as} $\bm{C^\infty}$ (one-form).
    \item Lemma 2.1.18: Let $U$ be an open subset of $\R^n$. If $f:U\to\R$ is a $C^\infty$ function, then
    \begin{equation*}
        \dd f = \sum_{i=1}^n\pdv{f}{x_i}\dd{x_i}
    \end{equation*}
    \item \textbf{Interior product} (of $\bm{v}$ with $\omega$): The function which combines a point $p\in U$, the vector $\bm{v}(p)\in T_p\R^n$, and the functional $\omega_p\in T_p^*\R^n$ to yield a real number. \emph{Denoted by} $\bm{\iota_{\pmb{v}(p)}\omega_p}$.
    \item Examples.
    \begin{itemize}
        \item If
        \begin{align*}
            \bm{v} &= \sum_{i=1}^ng_i\pdv{x_i}&
            \omega &= \sum_{i=1}^nf_i\dd x_i
        \end{align*}
        then
        \begin{equation*}
            \iota_{\bm{v}}\omega = \sum_{i=1}^nf_ig_i
        \end{equation*}
        \begin{itemize}
            \item We use multiplication and the fact that $(\dd x_i)_p(\pdv*{x_j})_p=\delta_{ij}$ to obtain this result.
            \item If $\bm{v},\omega\in C^\infty$, so is $\iota_{\bm{v}}\omega$, where $C^\infty$ refers to three different sets of smooth objects (vector fields, one-forms, and functions, respectively\footnote{Technically, these objects are all types of functions, though, so it is fair to call them all smooth.}).
        \end{itemize}
        \item As with $f$, if $\phi\in C^\infty(U)$, then
        \begin{equation*}
            \dd\phi = \sum_{i=1}^n\pdv{\phi}{x_i}\dd x_i
        \end{equation*}
        \item It follows if $\bm{v}$ is defined as in the first example that
        \begin{equation*}
            \iota_{\bm{v}}\dd\phi = \sum_{i=1}^ng_i\pdv{\phi}{x_i}
            = L_{\bm{v}}\phi
        \end{equation*}
    \end{itemize}
    \item \textbf{Integral curve} (of $\bm{v}$): A $C^1$ curve $\gamma:(a,b)\to U$ such that for all $t\in(a,b)$,
    \begin{equation*}
        \bm{v}(\gamma(t)) = (\gamma(t),{\textstyle\dv{\gamma}{t}}(t))
    \end{equation*}
    where $U\subset\R^n$ is open and $\bm{v}$ is a vector field on $U$.
    \begin{itemize}
        \item An equivalent condition if $\bm{v}=\sum_{i=1}^ng_i\dv*{x_i}$ and $g:U\to\R^n$ is defined by $(g_1,\dots,g_n)$ is that $\gamma$ satisfies the system of differential equations
        \begin{equation*}
            \dv{\gamma}{t} = g(\gamma(t))
        \end{equation*}
    \end{itemize}
    \item Theorem 2.2.4 (existence of integral curves): Let $U\subset\R^n$ open, $\bm{v}$ a vector field on $U$. If $p_0\in U$ and $a\in\R$, then there exist $I=(a-T,a+T)$ for some $T\in\R$, $U_0=N_r(p_0)\subset U$, and $\gamma_p:I\to U$ such that $\gamma_p(a)=p$ for all $p\in U_0$.
    \item Theorem 2.2.5 (uniqueness of integral curves): Let $U\subset\R^n$ open, $\bm{v}$ a vector field on $U$, and $\gamma_1:I_1\to U$ and $\gamma_2:I_2\to U$ integral curves for $\bm{v}$. If $a\in I_1\cap I_2$ and $\gamma_1(a)=\gamma_2(a)$, then
    \begin{equation*}
        \gamma_1|_{I_1\cap I_2} = \gamma_2|_{I_1\cap I_2}
    \end{equation*}
    and the curve $\gamma:I_1\cup I_2\to U$ defined by
    \begin{equation*}
        \gamma(t) =
        \begin{cases}
            \gamma_1(t) & t\in I_1\\
            \gamma_2(t) & t\in I_2
        \end{cases}
    \end{equation*}
    is an integral curve for $\bm{v}$.
    \item Theorem 2.2.6 (smooth dependence on initial data): Let $V\subset U\subset\R^n$ open, $\bm{v}$ a $C^\infty$-vector field on $V$, $I\subset\R$ an open interval, and $a\in I$. Let $h:V\times I\to U$ have the following properties.
    \begin{enumerate}
        \item $h(p,a)=p$.
        \item For all $p\in V$, the curve $\gamma_p:I\to U$ defined by $\gamma_p(t)=h(p,t)$ is an integral curve of $\bm{v}$.
    \end{enumerate}
    Then $h\in C^\infty$.
    \item \textbf{Autonomous} (system of ODEs): A system of ODEs that does not explicitly depend on the independent variable.
    \item $\dv*{\gamma}{t}=g(\gamma(t))$ is autonomous since $g$ does not depend on $t$.
    \item Theorem 2.2.7: Let $I=(a,b)$. For all $c\in\R$, define $I_c=(a-c,b-c)$. If $\gamma:I\to U$ is an integral curve, then the reparameterized curve $\gamma_c:I_c\to U$ defined by
    \begin{equation*}
        \gamma_c(t) = \gamma(t+c)
    \end{equation*}
    is an integral curve.
    \begin{itemize}
        \item Note that this is truly just a reparameterization; we still have, for instance,
        \begin{align*}
            \gamma_c(a-c) &= \gamma(a-c+c) = \gamma(a)&
            \gamma_c(b-c) &= \gamma(b-c+c) = \gamma(b)
        \end{align*}
    \end{itemize}
    \item \textbf{Integral} (of the system $\dv*{\gamma}{t}=g(\gamma(t))$): A $C^1$-function $\phi:U\to\R$ such that for every integral curve $\gamma(t)$, the function $t\mapsto\phi(\gamma(t))$ is constant.
    \begin{itemize}
        \item An alternate condition is that for all $t$,
        \begin{equation*}
            0 = \dv{t}\phi(\gamma(t))
            = (D\phi)_{\gamma(t)}\left( \dv{y}{t} \right)
            = (D\phi)_{\gamma(t)}(v)
            = L_v\phi(p)
        \end{equation*}
        where $\bm{v}(p)=(p,v)$.
    \end{itemize}
    \item Theorem 2.2.9: Let $U\subset\R^n$ open, $\phi\in C^1(u)$. Then $\phi$ is an integral of the system $\dv*{\gamma}{t}=g(\gamma(t))$ iff $L_{\bm{v}}\phi=0$.
    \item \textbf{Complete} (vector field): A vector field $\bm{v}$ on $U$ such that for every $p\in U$, there exists an integral curve $\gamma:\R\to U$ with $\gamma(0)=p$.
    \begin{itemize}
        \item Alternatively, for every $p$, there exists an integral curve that starts at $p$ and exists for all time.
    \end{itemize}
    \item \textbf{Maximal} (integral curve): An integral curve $\gamma:[0,b)\to U$ with $\gamma(0)=p$ such that it cannot be extended to an interval $[0,b')$ with $b'>b$.
    \item For a maximal curve, either\dots
    \begin{enumerate}
        \item $b=+\infty$;
        \item $|\gamma(t)|\to +\infty$ as $t\to b$;
        \item The limit set of $\{\gamma(t)\mid 0\leq t<b\}$ contains points on the boundary of $U$.
    \end{enumerate}
    \item Eliminating 2 and 3, as can be done with the following lemma, provides a means of proving that $\gamma$ exists for all positive time.
    \item Lemma 2.2.11: The scenarios 2 and 3 above cannot happen if there exists a proper $C^1$-function $\phi:U\to\R$ with $L_{\bm{v}}\phi=0$.
    \begin{proof}
        Suppose there exists $\phi\in C^1$ such that $L_{\bm{v}}\phi=0$. Then $\phi$ is constant on $\gamma(t)$ (say with value $c\in\R$) by definition. But then since $\{c\}\subset\R$ is compact and $\phi\in C^1$, $\phi^{-1}(c)\subset U$ is compact and, importantly, contains $\im(\gamma)$. The compactness of this set implies that $\gamma$ can neither "run off to infinity" as in scenario 2 or "run off the boundary" as in scenario 3.
    \end{proof}
    \item Theorem 2.2.12: If there exists a proper $C^1$-function $\phi:U\to\R$ with the property $L_{\bm{v}}\phi=0$, then the vector field $\bm{v}$ is complete.
    \begin{proof}
        Apply a similar argument to the interval $(-b,0]$ and join the two results.
    \end{proof}
    \item Example: Let $U=\R^2$ and let $\bm{v}$ be the vector field
    \begin{equation*}
        \bm{v} = x^3\pdv{y}-y\pdv{x}
    \end{equation*}
    Then $\phi(x,y)=2y^2+x^4$ is a proper function with the above property.
    \begin{itemize}
        \item Note that indeed, as per Theorem 2.2.12, we have that
        \begin{align*}
            L_{\bm{v}}\phi &= x^3\pdv{\phi}{y}-y\pdv{\phi}{x}\\
            &= x^3\cdot 4y-y\cdot 4x^3\\
            &= 0
        \end{align*}
    \end{itemize}
    \item We now build up to an alternate completeness condition (Theorem 2.2.15).
    \item \textbf{Support} (of $\bm{v}$): The following set. \emph{Denoted by} $\bm{\supp(\pmb{v})}$. \emph{Given by}
    \begin{equation*}
        \supp(\bm{v}) = \overline{\{q\in U\mid\bm{v}(q)\neq 0\}}
    \end{equation*}
    \item \textbf{Compactly supported} (vector field $\bm{v}$): A vector field $\bm{v}$ for which $\supp(\bm{v})$ is compact.
    \item Theorem 2.2.15: If $\bm{v}$ is compactly supported, then $\bm{v}$ is complete.
    \begin{proof}
        Let $p\in U$ be such that $\bm{v}(p)=0$. Define $\gamma_0:(-\infty,\infty)\to U$ by $\gamma_0(t)=p$ for all $t\in(-\infty,\infty)$. Since
        \begin{equation*}
            \dv{\gamma_0}{t} = 0 = \bm{v}(p) = \bm{v}(\gamma(t))
        \end{equation*}
        we know that $\gamma_0$ is an integral curve of $\bm{v}$.\par
        Now consider an arbitrary integral curve $\gamma:(-a,b)\to U$ having the property $\gamma(t_0)=p$ for some $t_0\in(-a,b)$. It follows by Theorem 2.2.5 that $\gamma$ and $\gamma_0$ coincide on the interval $(-a,a)$.\par
        By hypothesis, $\supp(\bm{v})$ is compact. Basic set theory tells us that for $\gamma$ arbitrary, either $\gamma(t)\in\supp(\bm{v})$ for all $t$ or there exists $t_0$ such that $\gamma(t_0)\in U\setminus\supp(\bm{v})$. But then by the definition of $\supp(\bm{v})$, $\bm{v}(\gamma(t_0))=0$. Thus, letting $p=\gamma(t_0)$, we have an associated $\gamma_0$ that $\gamma$ "runs along" while outside the support. It follows that in either case, $\gamma$ cannot go off to $\infty$ or go off the boundary of $U$ as $t\to b$.
    \end{proof}
    \item \textbf{Bump function}: A function $f:\R^n\to\R$ which is both smooth and compactly supported.
    \item $\bm{C_0^\infty(\pmb{\R}^n)}$: The vector space of all bump functions with domain $\R^n$.
    \item An application of Theorem 2.2.15.
    \begin{itemize}
        \item Suppose $\bm{v}$ is a vector field on $U$ and we want to inspect the integral curves of $\bm{v}$ on some $A\subset U$ compact. Let $\rho\in C_0^\infty(U)$ be such that $\rho(p)=1$ for all $p\in N_r(A)$, where $N_r(A)$ is some neighborhood of the set $A$. Then the vector field $\bm{w}=\rho\bm{v}$ is compactly supported and hence complete. However, it is also identical to $\bm{v}$ on $A$, so its integral curves on $A$ coincide with those of $\bm{v}$ on $A$.
    \end{itemize}
    \item $\bm{f_t}$: The map from $U\to U$ defined as follows, where $\bm{v}$ is complete. \emph{Given by}
    \begin{equation*}
        f_t(p) = \gamma_p(t)
    \end{equation*}
    \begin{itemize}
        \item Note that it is the fact that $\bm{v}$ is complete that justifies the existence of an integral curve $\gamma_p:\R\to U$ with $\gamma_p(0)=p$ for all $p\in U$.
    \end{itemize}
    \item Properties of $f_t$.
    \begin{enumerate}
        \item $\bm{v}\in C^\infty$ implies $f_t\in C^\infty$.
        \begin{proof}
            By Theorem 2.2.6.
        \end{proof}
        \item $f_0=\id_U$.
        \begin{proof}
            We have
            \begin{equation*}
                f_0(p) = \gamma_p(0) = p = \id_U(p)
            \end{equation*}
            as desired.
        \end{proof}
        \item $f_t\circ f_a=f_{t+a}$.
        \begin{proof}
            Let $q=f_a(p)$. Since $\bm{v}$ is complete and $q\in U$, there exists $\gamma_q$ such that $\gamma_q(0)=q$. It follows that $\gamma_p(a)=f_a(p)=q=\gamma_q(0)$. Thus, by Theorem 2.2.7, $\gamma_q(t)$ and $\gamma_p(t+a)$ are both integral curves of $\bm{v}$ with the same initial point. Therefore,
            \begin{equation*}
                (f_t\circ f_a)(p) = f_t(q)
                = \gamma_q(t)
                = \gamma_p(t+a)
                = f_{t+a}(p)
            \end{equation*}
            for all $t$, as desired.
        \end{proof}
        \item $f_t\circ f_{-t}=\id_U$.
        \begin{proof}
            See properties 2 and 3.
        \end{proof}
        \item $f_{-t}=f_t^{-1}$.
        \begin{proof}
            See property 4.
        \end{proof}
    \end{enumerate}
    \item Thus, $f_t$ is a $C^\infty$ \textbf{diffeomorphism}.
    \begin{itemize}
        \item "Hence, if $\bm{v}$ is complete, it generates a \textbf{one-parameter group} $f_t$ ($-\infty<t<\infty$) of $C^\infty$-diffeomorphisms of $U$" \parencite[40]{bib:DifferentialForms}.
    \end{itemize}
    \item \textbf{Diffeomorphism}: An isomorphism of smooth manifolds. In particular, it is an invertible function that maps one differentiable manifold to another such that both the function and its inverse are differentiable.
    \item \textbf{One-parameter group}: A continuous group homomorphism $\varphi:\R\to G$ from the real line $\R$ (as an additive group) to some other topological group $G$.
    \item If $\bm{v}$ is not complete, there is an analogous result, but it is trickier to formulate.
    \item \textbf{$\bm{f}$-related} (vector fields $\bm{v},\bm{w}$): Two vector fields $\bm{v},\bm{w}$ such that
    \begin{equation*}
        \dd f_p(\bm{v}(p)) = \bm{w}(f(p))
    \end{equation*}
    for all $p\in U$, where $\bm{v}$ is a $C^\infty$-vector field on $U\subset\R^n$ open, $\bm{w}$ is a $C^\infty$-vector field on $W\subset\R^m$ open, and $f:U\to W$ is a $C^\infty$ map.
    \begin{itemize}
        \item An alternate formulation is that in terms of coordinates,
        \begin{equation*}
            w_i(q) = \sum_{j=1}^n\pdv{f_i}{x_j}(p)v_j(p)
        \end{equation*}
        where
        \begin{align*}
            \bm{v} &= \sum_{i=1}^nv_i\pdv{x_i}
            \bm{w} &= \sum_{j=1}^mw_j\pdv{y_i}
        \end{align*}
        for $v_i\in C^k(U)$ and $w_j\in C^k(W)$.
    \end{itemize}
    \item If $m=n$ and $f$ is a $C^\infty$ diffeomorphism, then $\bm{w}$ is the vector field defined by the equation
    \begin{equation*}
        w_i = \sum_{j=1}^n\left( \pdv{f_i}{x_j}v_j \right)\circ f^{-1}
    \end{equation*}
    \item Theorem 2.2.18: If $f:U\to W$ is a $C^\infty$ diffeomorphism and $\bm{v}$ is a $C^\infty$ vector field on $U$, then there exists a unique $C^\infty$ vector field $\bm{w}$ on $W$ having the property that $\bm{v}$ and $\bm{w}$ are $f$-related.
    \begin{proof}
        See the above.
    \end{proof}
    \item \textbf{Pushforward} (of $\bm{v}$ by $f$): The vector field $\bm{w}$ shown to exist by Theorem 2.2.18. \emph{Denoted by} $\bm{f_*\pmb{v}}$.
    \item Theorem 2.2.20: Let $U_1,U_2\subset\R^n$ open, $\bm{v}_1,\bm{v}_2$ vector fields on $U_1,U_2$, and $f:U_1\to U_2$ a $C^\infty$ map. If $\bm{v}_1,\bm{v}_2$ are $f$-related, every integral curve $\gamma:I\to U_1$ of $\bm{v}_1$ gets mapped by $f$ onto an integral curve $f\circ\gamma:I\to U_2$ of $\bm{v}_2$.
    \begin{proof}
        We want to show that
        \begin{equation*}
            \bm{v}_2((f\circ\gamma)(t)) = \left( (f\circ\gamma)(t),\eval{\textstyle\dv{t}(f\circ\gamma)}_t \right)
        \end{equation*}
        We are given that
        \begin{align*}
            \bm{v}_1(\gamma(t)) &= \left( \gamma(t),\eval{\textstyle\dv{\gamma}{t}}_t \right)&
            \dd f_p(\bm{v}_1(p)) &= \bm{v}_2(f(p))
        \end{align*}
        Let $p=\gamma(t)$ and $q=f(p)$. Then
        \begin{align*}
            \bm{v}_2((f\circ\gamma)(t)) &= \bm{v}_2(f(p))\\
            &= \dd f_p(\bm{v}_1(p))\\
            &= \dd f_p(\bm{v}_1(\gamma(t)))\\
            &= \dd f_p\left( \gamma(t),\eval{\textstyle\dv{\gamma}{t}}_t \right)\\
            &= \dd f_p\left( p,\eval{\textstyle\dv{\gamma}{t}}_t \right)\\
            &= \left( q,Df(p)\left( \eval{\textstyle\dv{\gamma}{t}}_t \right) \right)\\
            % &= \left( f(\gamma(t)),\dd f_p\left( \eval{\textstyle\dv{\gamma}{t}}_t \right) \right)\\
            &= \left( (f\circ\gamma)(t),\eval{\textstyle\dv{t}(f\circ\gamma)}_t \right)
        \end{align*}
        as desired.
    \end{proof}
    \item Corollary 2.2.21: In the setting of Theorem 2.2.20, suppose $\bm{v}_1,\bm{v}_2$ are complete. Let $(f_{i,t})_{t\in\R}:U_i\to U_i$ be the one-parameter group of diffeomorphisms generated by $\bm{v}_i$. Then
    \begin{equation*}
        f\circ f_{1,t} = f_{2,t}\circ f
    \end{equation*}
    \begin{proof}
        We have that
        \begin{align*}
            (f\circ f_{1,t})(p) &= (f\circ\gamma_p)(t)
        \end{align*}
        By Theorem 2.2.20, the above is an integral curve of $\bm{v}_2$. Let $f(p)=q$. Then
        \begin{align*}
            (f_{2,t}\circ f)(p) &= f_{2,t}(q)\\
            &= \gamma_q(t)
        \end{align*}
        ...\par
        \textcite{bib:DifferentialForms} proves that if $\phi\in C^\infty(U_2)$ and $f^*\phi=\phi\circ f$, then
        \begin{equation*}
            f^*L_{\bm{v}_2}\phi = L_{\bm{v}_1}f^*\phi
        \end{equation*}
        by virtue of the observations that if $f(p)=q$, then at the point $p$, the right-hand side above is $(\dd\phi)_q\circ\dd f_p(\bm{v}_1(p))$ by the chain rule and by definition the left hand side is $\dd\phi_q(\bm{v}_2(q))$. Moreover, by definition, $\bm{v}_2(q)=\dd f_p(\bm{v}_1(p))$ so the two sides are the same.
    \end{proof}
    \item Theorem 2.2.22: For $i=1,2,3$, let $U_i\subset\R^{n_i}$ open and $\bm{v}_i$ a vector field on $U_i$. For $i=1,2$, let $f_i:U_i\to U_{i+1}$ be a $C^\infty$ map. If $\bm{v}_1,\bm{v}_2$ are $f_1$-related and $\bm{v}_2,\bm{v}_3$ are $f_2$-related, then $\bm{v}_1,\bm{v}_3$ are $(f_2\circ f_1)$-related. In particular, if $f_1,f_2$ are diffeomorphisms, we have
    \begin{equation*}
        (f_2)_*(f_1)_*\bm{v}_1 = (f_2\circ f_1)_*\bm{v}_1
    \end{equation*}
    \item \textbf{Pullback} (of $\mu$ on $U$): The function from $U\to T_p^*\R^n$ defined as follows, where $U\subset\R^n$ and $V\subset\R^m$ are open, $f:U\to V$ is a $C^\infty$ map, and $\mu$ is a one-form on $V$. \emph{Denoted by} $\bm{f^*\mu}$. \emph{Given by}
    \begin{equation*}
        p \mapsto \mu_{f(p)}\circ\dd{f_p}
    \end{equation*}
    \item If $\phi:V\to\R$ is a $C^\infty$ map and $\mu=\dd{\phi}$, then
    \begin{equation*}
        \mu_q\circ\dd f_p = \dd\phi_q\circ\dd f_p
        = \dd(\phi\circ f)_p
    \end{equation*}
    \begin{itemize}
        \item In other words,
        \begin{equation*}
            f^*\mu = \dd\phi\circ f
        \end{equation*}
    \end{itemize}
    \item Theorem 2.2.24: If $\mu$ is a $C^\infty$ one-form on $V$, its pullback $f^*\mu$ is $C^\infty$.
    \begin{proof}
        See Exercise 2.2.ii.
    \end{proof}
\end{itemize}




\end{document}