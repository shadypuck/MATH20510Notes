\documentclass[../notes.tex]{subfiles}

\pagestyle{main}
\renewcommand{\chaptermark}[1]{\markboth{\chaptername\ \thechapter\ (#1)}{}}

\begin{document}




\chapter{Multilinear Algebra}
\section{Notes}
\begin{itemize}
    \item \marginnote{3/28:}Motivation for the course and an overview of \textcite{bib:DifferentialForms}.
    \item \marginnote{3/30:}Plan:
    \begin{itemize}
        \item More (multi)linear algebra.
    \end{itemize}
    \item Dual spaces.
    \item Let $V$ be an $n$-dimensional real vector space.
    \item $\bm{\Hom(V,\pmb{\R})}$: The set of all homomorphisms (i.e., linear maps) from $V$ to $\R$. \emph{Also known as} $\bm{V^*}$.
    \item \textbf{Dual basis} (for $V^*$): The set of linear transformations from $V$ to $\R$ defined by
    \begin{equation*}
        \vec{e}_j \mapsto
        \begin{cases}
            1 & j=i\\
            0 & j\neq i
        \end{cases}
    \end{equation*}
    where $\vec{e}_1,\dots,\vec{e}_n$ is a basis of $V$. \emph{Denoted by} $\vec{e}_1^*,\dots,\vec{e}_n^*$.
    \item Check: $\vec{e}_1^*,\dots,\vec{e}_n^*$ are a basis for $V^*$.
    \begin{itemize}
        \item Are they linearly independent? Let $c_1\vec{e}_1^*+\cdots+c_n\vec{e}_n^*=0\in\Hom(V,\R)$. Then
        \begin{equation*}
            c_i = (c_1\vec{e}_1^*+\cdots+c_n\vec{e}_n^*)(\vec{e}_i) = 0\in\R
        \end{equation*}
        as desired.
        \item Span? Let $\varphi\in\Hom(V,\R)$. Then we can verify that
        \begin{equation*}
            \varphi(\vec{e}_1)\vec{e}_1^*+\cdots+\varphi(\vec{e}_n)\vec{e}_n^* = \varphi
        \end{equation*}
        \begin{itemize}
            \item We prove this by verifying the previous statement on the basis of $V$ (if two linear transformations have the same action on the basis of a vector space, they are equal).
        \end{itemize}
    \end{itemize}
    \item With a choice of basis for $V$, we obtain an isomorphism $\varepsilon:V\to V^*$ with the mapping $\vec{e}_i\mapsto\vec{e}_i^*$ for all $i$.
    \item The dual space is known as such because $(V^*)^*\cong V$, where $\cong$ is \textbf{canonical} (no choice of basis is needed).
    \item One more property of dual spaces: \textbf{functoriality}.
    \begin{itemize}
        \item Given a linear transformation $A:V\to W$, we know that $A^*:W^*\to V^*$ where $A^*$ is the transpose of $A$. In particular, if $\varphi\in W^*$, then $\varphi\circ A:V\to\R$.
        \item Claim: $A^*$ is linear.
    \end{itemize}
    \item \textbf{Functoriality}: If $A:V\to W$ and $B:W\to U$, then $B^*:U^*\to W^*$ and $A^*:W^*\to V^*$. The functoriality statement is that $(B\circ A)^*=A^*\circ B^*$.
    \item $A^*$ is the \textbf{pullback} (or transpose) of $A$.
    \item Let $\vec{v}_1,\dots,\vec{v}_n$ be a basis for $V$ and $\vec{w}_1,\dots,\vec{w}_m$ be a basis for $W$. Then $[A]_{\vec{v}_1,\dots,\vec{v}_n}^{\vec{w}_1,\dots,\vec{w}_m}=A$ is the matrix of the linear transformation $A$ with respect to these bases. Then if $\vec{v}_1^*,\dots,\vec{v}_n^*$ and $\vec{w}_1^*,\dots,\vec{w}_m^*$ are the corresponding dual bases, then $[A^*]_{\vec{v}_1^*,\dots,\vec{v}_n^*}^{\vec{w}_1^*,\dots,\vec{w}_m^*}=A^T$. We can and should verify this for ourselves.
    \item This is over the real numbers, so $A^*$ is just the transpose because there are no complex numbers of which to take the conjugate!
    \item A generalization: Tensors.
    \item \textbf{$\bm{k}$-tensor}: A \textbf{multilinear} map
    \begin{equation*}
        T:\underbrace{V\times\cdots\times V}_{k\text{ times}}\to\R
    \end{equation*}
    \item \textbf{Multilinear} (map $T$): A function $T$ such that
    \begin{align*}
        T(\vec{v}_1,\dots,\vec{v}_i^1+\vec{v}_i^2,\dots,\vec{v}_k) &= T(\vec{v}_1,\dots,\vec{v}_i^1,\dots,\vec{v}_k)+T(\vec{v}_1,\dots,\vec{v}_i^2,\dots,\vec{v}_k)\\
        T(\vec{v}_1,\dots,\lambda \vec{v}_i,\dots,\vec{v}_k) &= \lambda T(\vec{v}_1,\dots,\vec{v}_i,\dots,\vec{v}_k)
    \end{align*}
    for all $(\vec{v}_1,\dots,\vec{v}_k)\in V^k$.
    \item The determinant is an $n$-tensor!
    \item 1-tensors are just covectors.
    \item $\bm{L^k(V)}$: The vector space of all $k$-tensors on $V$.
    \item Calculating $\dim L^k(V)$. (Answer not given in this class.)
    \item Let $A:V\to W$. Then $A^*:L^k(W)\to L^k(V)$.
    \begin{itemize}
        \item Check $(A\circ B)^*=B^*\circ A^*$.
    \end{itemize}
    \item \textbf{Multi-index of $\bm{n}$ of length $\bm{k}$}: A $k$-tuple $(i_1,\dots,i_k)$ where each $i_j\in\N$ satisfies $1\leq i_j\leq n$ ($j=1,\dots,k$). \emph{Denoted by} $\bm{I}$.
    \item Let $\vec{e}_1,\dots,\vec{e}_n$ be a basis for $V$.
    \item \textbf{Tensor product} (of $T_1\in L^k(V)$, $T_2\in L^l(V)$): The function from $V^{k+l}$ to $\R$ defined by
    \begin{equation*}
        (\vec{v}_1,\dots,\vec{v}_{k+l}) \mapsto T_1(\vec{v}_1,\dots,\vec{v}_k)T_2(\vec{v}_{k+1},\dots,\vec{v}_{k+l})
    \end{equation*}
    \emph{Denoted by} $\bm{T_1\otimes T_2}$.
    \item Claims:
    \begin{enumerate}
        \item $T_1\otimes T_2\in L^{k+l}(V)$.
        \item $A^*(T_1\otimes T_2)=A^*(T_1)\otimes A^*(T_2)$.
    \end{enumerate}
    \item $\bm{\vec{e}_I^*}$: The function $\vec{e}_{i_1}^*\otimes\cdots\otimes\vec{e}_{i_k}^*$, where $I=(i_1,\dots,i_k)$ is a multi-index of $n$ of length $k$.
    \item Claim: Letting $I$ range over all $n^k$ multi-indices of $n$ of length $k$, the $\vec{e}_I^*$ are a basis for $L^k(V)$.
    \item If $V=\R$, then $V=\R\vec{e}_1$. If $V=\R^2$, then $V=\R\vec{e}_1\oplus\R\vec{e}_2$.
    \item We know that $L^1(V)=V^*=R\vec{e}_1^*$. Thus, $\vec{e}_1^*\otimes\vec{e}_2^*:V\times V\to\R$. Thus, for example,
    \begin{equation*}
        (\vec{e}_1^*\otimes\vec{e}_2^*)((1,2),(3,4)) = \vec{e}_1^*(1,2)\cdot \vec{e}_2^*(3,4)
        = 1\cdot 4
        = 4
    \end{equation*}
    \item \marginnote{4/1:}Plan: More multilinear algebra.
    \begin{itemize}
        \item Properties of the tensor product.
        \item Sign of a permutation.
        \item Alternating tensors (lead into differential forms down the road).
    \end{itemize}
    \item Recall: $V$ is an $n$-dimensional vector space over $\R$ with basis $e_1,\dots,e_n$. $\lin[k]{V}$ is the vector space of $k$-tensors on $V$. $\{e_I^*\mid I\text{ a multiindex of }n\text{ of length }k\}$ is a basis for $\lin[k]{V}$.
    \item For example, if $V=\R^2$ and $T\in\lin[2]{V}$, then
    \begin{equation*}
        T(a_1e_1+a_2e_2,b_1e_1+b_2e_2) = a_1b_1T(e_1,e_1)+a_1b_2T(e_1,e_2)+a_2b_1T(e_2,e_1)+a_2b_2T(e_2,e_2)
    \end{equation*}
    \begin{itemize}
        \item A basis of $\lin[2]{V}$ is
        \begin{equation*}
            \{e_1^*\otimes e_1^*,e_1^*\otimes e_2^*,e_2^*\otimes e_1^*,e_2^*\otimes e_2^*\}
        \end{equation*}
        \item Recall that some basic properties are
        \begin{align*}
            e_1^*\otimes e_2^*((1,2),(3,4)) &= 1\cdot 4 = 4&
            e_2^*\otimes e_1^*((1,2),(3,4)) &= 2\cdot 3 = 6
        \end{align*}
        \item It follows by the initial decomposition of $T$ that
        \begin{equation*}
            T = a_1b_1e_1^*\otimes e_1^*+a_1b_2e_1^*\otimes e_2^*+a_2b_1e_2^*\otimes e_1^*+a_2b_2e_2^*\otimes e_2^*
        \end{equation*}
    \end{itemize}
    \item Important consequence: To know the action of $T$ on an arbitrary pair of vectors, you need only know its action on the basis; a higher-dimensional generalization of the earlier property.
    \item Note that
    \begin{equation*}
        e_I^*(e_J) = \delta_{IJ} =
        \begin{cases}
            1 & I=J\\
            0 & I\neq J
        \end{cases}
    \end{equation*}
    \item Basic properties of the tensor product.
    \begin{enumerate}
        \item \emph{Right-distributive}: If $T_1\in\lin[k]{V}$ and $T_2,T_3\in\lin[\ell]{V}$, then
        \begin{equation*}
            T_1\otimes(T_2+T_3) = T_1\otimes T_2+T_1\otimes T_3
        \end{equation*}
        \item \emph{Left-distributive}: If $T_1,T_2\in\lin[k]{V}$ and $T_3\in\lin[\ell]{V}$, then
        \begin{equation*}
            (T_1+T_2)\otimes T_3 = T_1\otimes T_3+T_2\otimes T_3
        \end{equation*}
        \item \emph{Associative}: If $T_1\in\lin[k]{V}$, $T_2\in\lin[\ell]{V}$, and $T_3\in\lin[m]{V}$, then
        \begin{equation*}
            T_1\otimes(T_2\otimes T_3) = (T_1\otimes T_2)\otimes T_2
            = T_1\otimes T_2\otimes T_3
        \end{equation*}
        \item \emph{Scalar multiplication}: If $T_1\in\lin[k]{V}$, $T_2\in\lin[\ell]{V}$, and $\lambda\in\R$, then
        \begin{equation*}
            (\lambda T_1)\otimes T_2 = \lambda(T_1\otimes T_2)
            = T_1\otimes(\lambda T_2)
        \end{equation*}
    \end{enumerate}
    \item Note that the tensor product is not commutative.
    \item Aside: Defining the sign of a permutation.
    \item $\bm{S_A}$: The set of all automorphisms of $A$ (bijections from $A$ to $A$), where $A$ is a set.
    \item $\bm{S_n}$: The set $S_{[n]}$.
    \item Given $\sigma_1,\sigma_2\in S_n$, $\sigma_1\circ\sigma_2\in S_n$.
    \begin{itemize}
        \item Thus, $S_n$ is a \textbf{group}.
    \end{itemize}
    \item \textbf{Transposition}: A function $\tau\in S_n$ such that
    \begin{equation*}
        \tau(k) =
        \begin{cases}
            j & k=i\\
            i & k=j\\
            k & k\neq i,j
        \end{cases}
    \end{equation*}
    for some $i,j\in[n]$. \emph{Denoted by} $\bm{\tau_{i,j}}$.
    \item Theorem: An element of $S_n$ can be written as the product of transpositions (i.e., for all $\sigma\in S_n$, there exist $\tau_1,\dots,\tau_m\in S_n$ such that $\sigma=\tau_1\circ\cdots\circ\tau_m$).
    \item \textbf{Sign} (of $\sigma\in S_n$): The number (mod 2) of transpositions whose product equals $\sigma$. \emph{Denoted by} $\bm{(-1)^\sigma}$, $\bm{\sgn(\sigma)}$.
    \item Theorem: The sign of $\sigma$ is well-defined. Additionally,
    \begin{equation*}
        (-1)^{\sigma_1\sigma_2} = (-1)^{\sigma_1}\cdot(-1)^{\sigma_2}
    \end{equation*}
    \item Example: Consider the identity permutation. $(-1)^\sigma=+1$. We can think of this as the product of zero transpositions or, for instance, as the product of the two transpositions $\tau_{1,2}\circ\tau_{1,2}$. Another example would be $\tau_{2,3}\circ\tau_{1,2}\circ\tau_{1,2}\circ\tau_{2,3}$.
    \item Theorem: Let $X_i$ be a rational or polynomial function for each $i\in\N$. Then
    \begin{equation*}
        (-1)^\sigma = \prod_{i<j}\frac{X_{\sigma(i)}-X_{\sigma(j)}}{X_i-X_j}
    \end{equation*}
    \item Example: For the permutation $\sigma=(1,2,3)$, we have
    \begin{align*}
        (-1)^\sigma &= \frac{X_{\sigma(1)}-X_{\sigma(2)}}{X_1-X_2}\cdot\frac{X_{\sigma(1)}-X_{\sigma(3)}}{X_1-X_3}\cdot\frac{X_{\sigma(2)}-X_{\sigma(3)}}{X_2-X_3}\\
        &= \frac{X_2-X_3}{X_1-X_2}\cdot\frac{X_2-X_1}{X_1-X_3}\cdot\frac{X_3-X_1}{X_2-X_3}\\
        &= \frac{-(X_1-X_2)}{X_1-X_2}\cdot\frac{-(X_1-X_3)}{X_1-X_3}\cdot\frac{X_2-X_3}{X_2-X_3}\\
        &= -1\cdot -1\cdot 1\\
        &= +1
    \end{align*}
    which squares with the fact that $\sigma=\tau_{1,2}\circ\tau_{2,3}$.
    \item Claims to verify with the above formula:
    \begin{enumerate}
        \item $\sgn(\sigma)\in\{\pm 1\}$.
        \item $\sgn(\tau_{i,j})=-1$.
        \item $\sgn(\sigma_1\sigma_2)=\sgn(\sigma_1)\sgn(\sigma_2)$.
    \end{enumerate}
\end{itemize}



\section{Chapter 1: Multilinear Algebar}
\emph{From \textcite{bib:DifferentialForms}.}
\begin{itemize}
    \item \marginnote{3/31:}\textcite{bib:DifferentialForms} defines real vector spaces, the operations on them, their basic properties, and the zero vector.
    \item \textbf{Linearly independent} (vectors $v_1,\dots,v_k$): A finite set of vectors $v_1,\dots,v_k\in V$ such that the map from $\R^k$ to $V$ defined by $(c_1,\dots,c_k)\mapsto c_1v_1+\cdots+c_kv_k$ is injective.
    \item \textbf{Spanning} (vectors $v_1,\dots,v_k$): We require that the above map is surjective.
    \item \textcite{bib:DifferentialForms} defines basis, finite-dimensional vector space, dimension, subspace, linear map, and kernel.
    \item \textbf{Image} (of $A:V\to W$): The range space of $A$, a subspace of $W$. \emph{Also known as} $\bm{\im(A)}$.
    \item \textcite{bib:DifferentialForms} defines the matrix of a linear map.
    \item \textbf{Inner product} (on $V$): A map $B:V\times V\to\R$ with the following three properties.
    \begin{itemize}
        \item \emph{Bilinearity}: For vectors $v,v_1,v_2,w\in V$ and $\lambda\in\R$, we have
        \begin{equation*}
            B(v_1+v_2,w) = B(v_1,w)+B(v_2,w)
        \end{equation*}
        and
        \begin{equation*}
            B(\lambda v,w) = \lambda B(v,w)
        \end{equation*}
        \item \emph{Symmetry}: For vectors $v,w\in V$, we have $B(v,w)=B(w,v)$.
        \item \emph{Positivity}: For every vector $v\in V$, we have $B(v,v)\geq 0$. Moreover, if $v\neq 0$, then $B(v,v)>0$.
    \end{itemize}
    \item \textbf{$\bm{W}$-coset}: A set of the form $\{v+w\mid w\in W\}$, where $W$ is a subspace $V$ and $v\in V$. \emph{Denoted by} $\bm{v+W}$.
    \begin{itemize}
        \item If $v_1-v_2\in W$, then $v_1+W=v_2+W$.
        \item It follows that the distinct $W$-cosets decompose $V$ into a disjoint collection of subsets of $V$.
    \end{itemize}
    \item \textbf{Quotient space} (of $V$ by $W$): The set of distinct $W$-cosets in $V$, along with the following definitions of vector addition and scalar multiplication.
    \begin{align*}
        (v_1+W)+(v_2+W) &= (v_1+v_2)+W&
        \lambda(v+W) &= (\lambda v)+W
    \end{align*}
    \emph{Denoted by} $\bm{V/W}$.
    \item \textbf{Quotient map}: The linear map $\pi:V\to V/W$ defined by
    \begin{equation*}
        \pi(v) = v+W
    \end{equation*}
    \begin{itemize}
        \item $\pi$ is surjective.
        \item Note that $\ker(\pi)=W$ since for all $w\in W$, $\pi(w)=w+W=0+W$, which is the zero vector in $V/W$.
    \end{itemize}
    \item If $V,W$ are finite dimensional, then
    \begin{equation*}
        \dim(V/W) = \dim(V)-\dim(W)
    \end{equation*}
    \item Proposition 1.2.9: Let $A:V\to U$ be a linear map. If $W\subset\ker(A)$, then there exists a unique linear map $A^\sharp:V/W\to U$ with the property that $A=A^\sharp\circ\pi$, where $\pi:V\to V/W$ is the quotient map.
    \begin{itemize}
        \item This proposition rephrases in terms of quotient spaces the fact that if $w\in W$, then $A(v+w)=Av$.
    \end{itemize}
    \item \textbf{Dual space} (of $V$): The set of all linear functions $\ell:V\to\R$, along with the following definitions of vector addition and scalar multiplication.
    \begin{align*}
        (\ell_1+\ell_2)(v) &= \ell_1(v)+\ell_2(v)&
        (\lambda\ell)(v) &= \lambda\cdot\ell(v)
    \end{align*}
    \emph{Denoted by} $V^*$.
    \item \textbf{Dual basis} (of $e_1,\dots,e_n$ a basis of $V$): The basis of $V^*$ consisting of the $n$ functions that take every $v=c_1e_1+\cdots+c_ne_n$ to one of the $c_i$. \emph{Denoted by} $\bm{e_1^*,...,e_n^*}$. \emph{Given by}
    \begin{equation*}
        e_i^*(v) = c_i
    \end{equation*}
    for all $v\in V$.
    \item Claim 1.2.12: If $V$ is an $n$-dimensional vector space with basis $e_1,\dots,e_n$, then $e_1^*,\dots,e_n^*$ is a basis of $V^*$.
    \begin{proof}
        We will first prove that $e_1^*,\dots,e_n^*$ spans $V^*$. Let $\ell\in V^*$ be arbitrary. Set $\lambda_i=\ell(e_i)$ for all $i\in[n]$. Define $\ell'=\sum_{i=1}^n\lambda_ie_i^*$. Then
        \begin{equation*}
            \ell'(e_j) = \sum_{i=1}^n\lambda_ie_i^*(e_j)
            = \lambda_j\cdot 1
            = \ell(e_j)
        \end{equation*}
        for all $j\in[n]$. Therefore, since $\ell,\ell'$ take identical values on the basis of $V$, $\ell=\ell'$, as desired.\par
        We now prove that $e_1^*,\dots,e_n^*$ spans $V^*$. Let $\sum_{i=1}^n\lambda_ie_i^*=0$. Then for all $j\in[n]$,
        \begin{equation*}
            \lambda_j = (\sum_{i=1}^n\lambda_ie_i^*)(e_j)
            = 0
        \end{equation*}
        as desired.
    \end{proof}
    \item \textbf{Transpose} (of $A$): The map from $W^*$ to $V^*$ defined by $\ell\mapsto A^*\ell=\ell\circ A$ for all $\ell\in W^*$.
    \item Claim 1.2.15: If $e_1,\dots,e_n$ is a basis of $V$, $f_1,\dots,f_m$ is a basis of $W$, $e_1^*,\dots,e_n^*$ and $f_1^*,\dots,f_m^*$ are the corresponding dual bases, and $[a_{i,j}]$ is the $m\times n$ matrix of $A$ with respect to $\{e_i\},\{f_i\}$, then the linear map $A^*$ is defined in terms of $\{f_i^*\},\{e_i^*\}$ by the transpose matrix $(a_{j,i})$.
\end{itemize}




\end{document}