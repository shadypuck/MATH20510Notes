\documentclass[../notes.tex]{subfiles}

\pagestyle{main}
\renewcommand{\chaptermark}[1]{\markboth{\chaptername\ \thechapter\ (#1)}{}}
\setcounter{chapter}{4}

\begin{document}




\chapter{Differentiation}
\section{Vector Calculus Operations}
\begin{itemize}
    \item \marginnote{4/27:}Announcements.
    \begin{itemize}
        \item No class this Friday, next Monday.
        \item Midterm next Friday.
        \begin{itemize}
            \item Up through Chapter 2.
            \item The exam will likely be computationally heavy.
            \item Compute $\dd$, pullbacks, interior products, Lie derivatives, etc.
            \item Emphasis on Chapter 2 as opposed to Chapter 1 even though it all builds on itself.
            \item He'll probably cook up a few problems too.
        \end{itemize}
        \item There is a recorded lecture for us.
        \begin{itemize}
            \item On Chapter 3 content.
            \item We'll cover Chapter 3 in kind of an impressionistic way as it is.
        \end{itemize}
        \item There are also some notes on the physics stuff.
    \end{itemize}
    \item Vector calculus operations.
    \begin{itemize}
        \item In one dimension, you have functions, and you take derivatives.
        \begin{itemize}
            \item The derivative operation does essentially map $\Omega^0\to\Omega^1$ or $C^\infty(\R)\to C^\infty(\R)$.
        \end{itemize}
        \item In two dimensions, ...
        \begin{itemize}
            \item $\dd^2=0$ reflects the fact that gradient vector fields are curl-free.
        \end{itemize}
        \item If you want to understand the 2D-curl business\dots
        \begin{itemize}
            \item $\crl(v):\R^2\to\R$ is intuitively about balls spinning around in a vector field.
            \item There's also a nice formula to compute it.
            \item And then there's a connection with $\dd:\Omega^1\to\Omega^2$.
        \end{itemize}
        \item In 3D, you can take top-dimensional forms (which are just functions) and bottom-dimensional forms (which are by definition functions) and you can work out an identification between them.
        \item Note that $\crl:\mathfrak{X}(\R^2)\to C^\infty(\R^2)$, where $\mathfrak{X}(\R^2)$ is the space of vector fields.
    \end{itemize}
    \item The musical operator $\sharp$ identifies forms with vector fields, i.e., $\sharp:\Omega^1\to\mathfrak{X}(\R^2)$.
    \item Properties of exterior derivatives $\dd:\ome[k]{U}\to\ome[k+1]{U}$.
    \begin{enumerate}
        \item $\dd{(\omega_1+\omega_2)}=\dd{\omega_1}+\dd{\omega_2}$ and $\dd{(\lambda\omega)}=\lambda\dd{\omega}$.
        \item Product rule $\dd{(\omega_1\wedge\omega_2)}=\dd{\omega_1}\wedge\omega_2+(-1)^k\omega_1\wedge\dd{\omega_2}$.
        \begin{itemize}
            \item Special case $k=\ell=0$. Then
            \begin{equation*}
                \dd{(fg)} = g\dd{f}+f\dd{g}
            \end{equation*}
            which is the usual product rule for gradient.
            \item Claim:
            \begin{equation*}
                \dd{\left( \sum_If_I\dd{x_I} \right)} = \sum_I\dd{f_I}\wedge\dd{x_I}
            \end{equation*}
            \begin{itemize}
                \item Let $\omega_1\in\Omega^k$ and $\omega_2\in\Omega^\ell$ be defined by
                \begin{align*}
                    \omega_1 &= \sum_If_I\dd{x_I}&
                    \omega_2 &= \sum_Jg_J\dd{x_J}
                \end{align*}
                where we're summing over all $I$ such that $|I|=k$ and all $J$ such that $|J|=\ell$. Then
                \begin{align*}
                    \omega_1\wedge\omega_2 &= \sum_{I,J}f_Ig_J\dd{x_I}\wedge\dd{x_J}
                    \dd{(\omega_1\wedge\omega_2)} &= \sum_{I,J}\dd{(f_Ig_J)}\wedge\dd{x_I}\wedge\dd{x_J}
                \end{align*}
                \item Note that
                \begin{equation*}
                    \dd{(f_Ig_J)} = g_J\dd{f_I}+f_I\dd{g_J}
                \end{equation*}
                and
                \begin{equation*}
                    \dd{g_J}\wedge\dd{x_I} = (-1)^k\dd{x_I}\wedge\dd{g_J}
                \end{equation*}
                \item These identities allow us to take the previous equation to
                \begin{align*}
                    \dd{(\omega_1\wedge\omega_2)} &= \sum_{I,J}g_J\dd{f_I}\wedge\dd{x_I}\wedge\dd{x_J}+(-1)^kf_I\dd{x_I}\wedge\dd{g_J}\wedge\dd{x_J}\\
                    &= \sum_{I,J}(\dd{f_I}\wedge\dd{x_I})\wedge(g_J\dd{x_J})+\sum_{I,J}(f_I\dd{x_I})\wedge(dd{g_J}\wedge\dd{x_J})\\
                    &= \dd{\omega_1}\wedge\omega_2+(-1)^k\omega_1\dd{\omega_2}
                \end{align*}
            \end{itemize}
        \end{itemize}
        \item $\dd^2=0$.
        \begin{itemize}
            \item Let $\omega=\sum_If_I\dd{x_I}$.
            \item Then
            \begin{align*}
                \dd^2(\omega) &= \dd(\dd{\omega})\\
                &= \dd(\sum_I\dd{f_I}\wedge\dd{x_I})\\
                &= \sum_I\dd(\dd{f_I}\wedge\dd{x_I})\tag*{Property 1}\\
                &= \sum_I\dd(\dd{f_I})\wedge\dd{x_I}\tag*{Property 2}
            \end{align*}
            so it suffices to just show that $\dd^2f=0$ for all $f\in\Omega^0$.
            \item We know that $\dd{f}=\sum_{i=1}^n\pdv*{f}{x_i}\dd{x_i}$. Thus,
            \begin{align*}
                \dd(\dd{f}) &= \sum_i\dd(\pdv{f}{x_i})\wedge\dd{x_i}\\
                &= \sum_{i,j}\pdv{f}{x_j}{x_i}\dd{x_j}\wedge\dd{x_i}\\
                &= 0
            \end{align*}
            \item The last equality holds because of commuting partial derivatives for smooth $f$, and the fact that changing order introduces a negative sign by some property.
        \end{itemize}
    \end{enumerate}
    \item In fact, if we fix $\dd^0:\ome[0]{U}\to\ome[1]{U}$ to be the "gradient," then these properties characterize the function $\dd$ on its domain and codomain. In particular, $\dd$ is the unique function on its domain and codomain that satisfies these properties.
    \begin{itemize}
        \item We define it by
        \begin{equation*}
            \dd{\left( \sum_If_I\dd{x_I} \right)} = \sum_I\dd{f_I}\wedge\dd{x_I}
        \end{equation*}
        \item The above properties characterize it axiomatically.
        \item We can prove this uniqueness theorem.
    \end{itemize}
    \item \textbf{Closed} (form): A form $\omega\in\ome[k]{U}$ such that $\dd{\omega}=0$.
    \item \textbf{Exact} (form): A form $\omega\in\ome[k]{U}$ such that $\omega=\dd{\eta}$ for some $\eta\in\ome[k-1]{U}$.
    \item $\dd^2=0$ implies closed and exact implies closed.
    \item \textbf{Poincar\'{e} lemma}: Locally closed forms are exact.
\end{itemize}



\section{Chapter 2: Differential Forms}
\emph{From \textcite{bib:DifferentialForms}.}
\begin{itemize}
    \item \marginnote{5/5:}As we formed the $k^\text{th}$ exterior powers $\lam[k]{V^*}$, we can form the $k^\text{th}$ exterior powers $\lam[k]{T_p^*\R^n}$.
    \begin{itemize}
        \item Since $\lam[1]{T_p^*\R^n}=T_p^*\R^n$, we can think of a one-form as a function which takes its value at $p$ in the space $\lam[1]{T_p^*\R^n}$.
    \end{itemize}
    \item \textbf{$\bm{k}$-form} (on $U$): A function which assigns to each point $p\in U$ an element $\omega_p\in\lam[k]{T_p^*\R^n}$, where $U\subset\R^n$ is open.
    \item We can use the wedge product to construct $k$-forms.
    \begin{itemize}
        \item Let $\omega_1,\dots,\omega_k$ be one-forms. Then $\omega_1\wedge\cdots\wedge\omega_k$ is the $k$-form whose value at $p$ is the wedge product
        \begin{equation*}
            (\omega_1\wedge\cdots\wedge\omega_k)_p = (\omega_1)_p\wedge\cdots\wedge(\omega_k)_p
        \end{equation*}
        \item Let $f_1,\dots,f_k$ be real-valued functions in $C^\infty(U)$. Suppose $\omega_i=\dd f_i$. Then we may obtain the $k$-form whose value at $p$ is
        \begin{equation*}
            (\dd f_1\wedge\cdots\wedge\dd f_k)_p = (\dd f_1)_p\wedge\cdots\wedge(\dd f_k)_p
        \end{equation*}
    \end{itemize}
    \item Since $(\dd x_1)_p,\dots,(\dd x_n)_p$ are a basis of $T_p^*\R^n$, the wedge products
    \begin{equation*}
        (\dd x_I)_p = (\dd x_{i_1})_p\wedge\cdots\wedge(\dd x_{i_k})_p
    \end{equation*}
    where $I=(i_1,\dots,i_k)$ is a strictly increasing multi-index of $n$ of length $k$ form a basis of $\lam[k]{T_p^*\R^n}$.
    \item Thus, every $\omega_p\in\lam[k]{T_p^*\R^n}$ has a unique decomposition
    \begin{equation*}
        \omega_p = \sum_Ic_I(\dd x_I)_p
    \end{equation*}
    where every $c_I\in\R$.
    \item Similarly, every $k$-form $\omega$ on $U$ has a unique decomposition
    \begin{equation*}
        \omega = \sum_If_I\dd{x_I}
    \end{equation*}
    where every $f_I:U\to\R$.
    \item \textbf{Class $\bm{C^r}$} ($k$-form): A $k$-form $\omega$ for which every $f_I$ in its decomposition is in $C^r(U)$.
    \item From here on out, we assume unless otherwise stated that all $k$-forms we consider are of class $C^\infty$.
    \item $\bm{\ome[k]{U}}$: The set of $k$-forms of class $C^\infty$ on $U$.
    \item $\bm{f\omega}$: The $k$-form defined as follows, where $f\in C^\infty(U)$ and $\omega\in\ome[k]{U}$. \emph{Given by}
    \begin{equation*}
        p \mapsto f(p)\omega_p
    \end{equation*}
    \item \textbf{Sum} (of $\omega_1,\omega_2$): The $k$-form defined as follows, where $\omega_1,\omega_2\in\ome[k]{U}$. \emph{Denoted by} $\bm{\omega_1+\omega_2}$. \emph{Given by}
    \begin{equation*}
        p \mapsto (\omega_1)_p+(\omega_2)_p
    \end{equation*}
    \item \textbf{Wedge product} (of $\omega_1,\omega_2$): The $(k_1+k_2)$-form defined as follows, where $\omega_1\in\ome[k_1]{U}$ and $\omega_2\in\ome[k_2]{U}$. \emph{Denoted by} $\bm{\omega_1\wedge\omega_2}$. \emph{Given by}
    \begin{equation*}
        p \mapsto (\omega_1)_p\wedge(\omega_2)_p
    \end{equation*}
    \item \textbf{Zero-form}: A function which assigns to each $p\in U$ an element of $\lam[0]{T_p^*\R^n}=\R$. \emph{Also known as} \textbf{real-valued function}.
    \item It follows from the definition of zero-forms that
    \begin{equation*}
        \ome[0]{U} = C^\infty(U)
    \end{equation*}
    \item \textbf{Exterior differentiation operation}: The operator from $\ome[0]{U}\to\ome[1]{U}$ which associates to a function $f\in C^\infty(U)$ the 1-form $\dd f$. \emph{Denoted by} $\mathbf{d}$.
    \item We now seek to define a generalized version of the exterior differentiation operation; in particular, we would like to define an analogous function $\dd:\ome[k]{U}\to\ome[k+1]{U}$.
    \item Desired properties of exterior differentiation.
    \begin{enumerate}
        \item If $\omega_1,\omega_2\in\ome[k]{U}$, then
        \begin{equation*}
            \dd(\omega_1+\omega_2) = \dd\omega_1+\dd\omega_2
        \end{equation*}
        \item If $\omega_1\in\ome[k]{U}$ and $\omega_2\in\ome[\ell]{U}$, then
        \begin{equation*}
            \dd(\omega_1\wedge\omega_2) = \dd\omega_1\wedge\omega_2+(-1)^k\omega_1\wedge\dd\omega_2
        \end{equation*}
        \item If $\omega\in\ome[k]{U}$, then
        \begin{equation*}
            \dd(\dd\omega) = 0
        \end{equation*}
    \end{enumerate}
    \item Consequences of these properties.
    \item Lemma 2.4.5: Let $U\subset\R^n$ open. If $f_1,\dots,f_k\in C^\infty(U)$, then
    \begin{equation*}
        \dd(\dd f_1\wedge\cdots\wedge\dd f_k) = 0
    \end{equation*}
    \begin{proof}
        We induct on $k$. For the base case $k=1$, we have that $\dd(\dd f_1)=0$ by Property 3. Now suppose inductively that we have proven the claim for $k-1$ functions; we now seek to prove it for $k$ functions. Let $\mu=\dd f_1\wedge\cdots\wedge\dd f_{k-1}$. Then by the induction hypothesis, $\dd\mu=0$. Therefore,
        \begin{align*}
            \dd(\dd f_1\wedge\cdots\wedge\dd f_k) &= \dd(\mu\wedge\dd f_k)\\
            &= \dd\mu\wedge\dd f_k+(-1)^{k-1}\mu\wedge\dd(\dd f_k)\tag*{Property 2}\\
            &= 0
        \end{align*}
        as desired.
    \end{proof}
    \item A special case of Lemma 2.4.5 is that
    \begin{equation*}
        \dd(\dd x_I) = 0
    \end{equation*}
    \item Now since every $k$-form $\omega\in\ome[k]{U}$ has a unique decomposition in terms of the $\dd x_I$, Property 2 and the above equation reveal that
    \begin{equation*}
        \dd\omega = \sum_I\dd f_I\wedge\dd x_I
    \end{equation*}
    \item Therefore, if there exists an operator $\dd$ satisfying Properties 1-3, then $\dd$ necessarily has the above form. All that's left is to show that the operator defined above has these properties.
    \item Proposition 2.4.10: Let $U\subset\R^n$ be open. There is a unique operator $\dd:\ome[*]{U}\to\ome[*+1]{U}$ satisfying Properties 1-3.
    \begin{proof}
        ...
    \end{proof}
    \item \textbf{Closed} ($k$-form): A $k$-form $\omega\in\ome[k]{U}$ for which $\dd\omega=0$.
    \item \textbf{Exact} ($k$-form): A $k$-form $\omega\in\ome[k]{U}$ such that $\omega=\dd\mu$ for some $\mu\in\ome[k-1]{U}$.
    \item Property 3 implies that every exact $k$-form is closed.
    \begin{itemize}
        \item The converse is not true even for 1-forms (see Exercise 2.1.iii).
        \item "It is a very interesting (and hard) question to determine if an open set $U$ has the following property: For $k>0$, every closed $k$-form is exact" \parencite[49]{bib:DifferentialForms}.
        \begin{itemize}
            \item Note that we do not consider zero-forms since there are no $(-1)$-forms for which to define exactness.
        \end{itemize}
    \end{itemize}
    \item If $f\in C^\infty(U)$ and $\dd f=0$, then $f$ is constant on connected components of $U$ (see Exercise 2.2.iii).
    \item Lemma 2.4.16 (Poincar\'{e} lemma): If $\omega$ is a closed form on $U$ of degree $k>0$, then for every point $p\in U$, there exists a neighborhood of $p$ on which $\omega$ is exact.
    \begin{proof}
        See Exercises 2.4.v and 2.4.vi.
    \end{proof}
\end{itemize}




\end{document}