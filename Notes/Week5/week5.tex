\documentclass[../notes.tex]{subfiles}

\pagestyle{main}
\renewcommand{\chaptermark}[1]{\markboth{\chaptername\ \thechapter\ (#1)}{}}
\setcounter{chapter}{4}

\begin{document}




\chapter{Differentiation}
\section{Vector Calculus Operations}
\begin{itemize}
    \item \marginnote{4/27:}Announcements.
    \begin{itemize}
        \item No class this Friday, next Monday.
        \item Midterm next Friday.
        \begin{itemize}
            \item Up through Chapter 2.
            \item The exam will likely be computationally heavy.
            \item Compute $\dd$, pullbacks, interior products, Lie derivatives, etc.
            \item Emphasis on Chapter 2 as opposed to Chapter 1 even though it all builds on itself.
            \item He'll probably cook up a few problems too.
        \end{itemize}
        \item There is a recorded lecture for us.
        \begin{itemize}
            \item On Chapter 3 content.
            \item We'll cover Chapter 3 in kind of an impressionistic way as it is.
        \end{itemize}
        \item There are also some notes on the physics stuff.
    \end{itemize}
    \item Vector calculus operations.
    \begin{itemize}
        \item In one dimension, you have functions, and you take derivatives.
        \begin{itemize}
            \item The derivative operation does essentially map $\Omega^0\to\Omega^1$ or $C^\infty(\R)\to C^\infty(\R)$.
        \end{itemize}
        \item In two dimensions, ...
        \begin{itemize}
            \item $\dd^2=0$ reflects the fact that gradient vector fields are curl-free.
        \end{itemize}
        \item If you want to understand the 2D-curl business\dots
        \begin{itemize}
            \item $\crl(v):\R^2\to\R$ is intuitively about balls spinning around in a vector field.
            \item There's also a nice formula to compute it.
            \item And then there's a connection with $\dd:\Omega^1\to\Omega^2$.
        \end{itemize}
        \item In 3D, you can take top-dimensional forms (which are just functions) and bottom-dimensional forms (which are by definition functions) and you can work out an identification between them.
        \item Note that $\crl:\mathfrak{X}(\R^2)\to C^\infty(\R^2)$, where $\mathfrak{X}(\R^2)$ is the space of vector fields.
    \end{itemize}
    \item The musical operator $\sharp$ identifies forms with vector fields, i.e., $\sharp:\Omega^1\to\mathfrak{X}(\R^2)$.
    \item Properties of exterior derivatives $\dd:\ome[k]{U}\to\ome[k+1]{U}$.
    \begin{enumerate}
        \item $\dd{(\omega_1+\omega_2)}=\dd{\omega_1}+\dd{\omega_2}$ and $\dd{(\lambda\omega)}=\lambda\dd{\omega}$.
        \item Product rule $\dd{(\omega_1\wedge\omega_2)}=\dd{\omega_1}\wedge\omega_2+(-1)^k\omega_1\wedge\dd{\omega_2}$.
        \begin{itemize}
            \item Special case $k=\ell=0$. Then
            \begin{equation*}
                \dd{(fg)} = g\dd{f}+f\dd{g}
            \end{equation*}
            which is the usual product rule for gradient.
            \item Claim:
            \begin{equation*}
                \dd{\left( \sum_If_I\dd{x_I} \right)} = \sum_I\dd{f_I}\wedge\dd{x_I}
            \end{equation*}
            \begin{itemize}
                \item Let $\omega_1\in\Omega^k$ and $\omega_2\in\Omega^\ell$ be defined by
                \begin{align*}
                    \omega_1 &= \sum_If_I\dd{x_I}&
                    \omega_2 &= \sum_Jg_J\dd{x_J}
                \end{align*}
                where we're summing over all $I$ such that $|I|=k$ and all $J$ such that $|J|=\ell$. Then
                \begin{align*}
                    \omega_1\wedge\omega_2 &= \sum_{I,J}f_Ig_J\dd{x_I}\wedge\dd{x_J}
                    \dd{(\omega_1\wedge\omega_2)} &= \sum_{I,J}\dd{(f_Ig_J)}\wedge\dd{x_I}\wedge\dd{x_J}
                \end{align*}
                \item Note that
                \begin{equation*}
                    \dd{(f_Ig_J)} = g_J\dd{f_I}+f_I\dd{g_J}
                \end{equation*}
                and
                \begin{equation*}
                    \dd{g_J}\wedge\dd{x_I} = (-1)^k\dd{x_I}\wedge\dd{g_J}
                \end{equation*}
                \item These identities allow us to take the previous equation to
                \begin{align*}
                    \dd{(\omega_1\wedge\omega_2)} &= \sum_{I,J}g_J\dd{f_I}\wedge\dd{x_I}\wedge\dd{x_J}+(-1)^kf_I\dd{x_I}\wedge\dd{g_J}\wedge\dd{x_J}\\
                    &= \sum_{I,J}(\dd{f_I}\wedge\dd{x_I})\wedge(g_J\dd{x_J})+\sum_{I,J}(f_I\dd{x_I})\wedge(dd{g_J}\wedge\dd{x_J})\\
                    &= \dd{\omega_1}\wedge\omega_2+(-1)^k\omega_1\dd{\omega_2}
                \end{align*}
            \end{itemize}
        \end{itemize}
        \item $\dd^2=0$.
        \begin{itemize}
            \item Let $\omega=\sum_If_I\dd{x_I}$.
            \item Then
            \begin{align*}
                \dd^2(\omega) &= \dd(\dd{\omega})\\
                &= \dd(\sum_I\dd{f_I}\wedge\dd{x_I})\\
                &= \sum_I\dd(\dd{f_I}\wedge\dd{x_I})\tag*{Property 1}\\
                &= \sum_I\dd(\dd{f_I})\wedge\dd{x_I}\tag*{Property 2}
            \end{align*}
            so it suffices to just show that $\dd^2f=0$ for all $f\in\Omega^0$.
            \item We know that $\dd{f}=\sum_{i=1}^n\pdv*{f}{x_i}\dd{x_i}$. Thus,
            \begin{align*}
                \dd(\dd{f}) &= \sum_i\dd(\pdv{f}{x_i})\wedge\dd{x_i}\\
                &= \sum_{i,j}\pdv{f}{x_j}{x_i}\dd{x_j}\wedge\dd{x_i}\\
                &= 0
            \end{align*}
            \item The last equality holds because of commuting partial derivatives for smooth $f$, and the fact that changing order introduces a negative sign by some property.
        \end{itemize}
    \end{enumerate}
    \item In fact, if we fix $\dd^0:\ome[0]{U}\to\ome[1]{U}$ to be the "gradient," then these properties characterize the function $\dd$ on its domain and codomain. In particular, $\dd$ is the unique function on its domain and codomain that satisfies these properties.
    \begin{itemize}
        \item We define it by
        \begin{equation*}
            \dd{\left( \sum_If_I\dd{x_I} \right)} = \sum_I\dd{f_I}\wedge\dd{x_I}
        \end{equation*}
        \item The above properties characterize it axiomatically.
        \item We can prove this uniqueness theorem.
    \end{itemize}
    \item \textbf{Closed} (form): A form $\omega\in\ome[k]{U}$ such that $\dd{\omega}=0$.
    \item \textbf{Exact} (form): A form $\omega\in\ome[k]{U}$ such that $\omega=\dd{\eta}$ for some $\eta\in\ome[k-1]{U}$.
    \item $\dd^2=0$ implies closed and exact implies closed.
    \item \textbf{Poincar\'{e} lemma}: Locally closed forms are exact.
\end{itemize}




\end{document}