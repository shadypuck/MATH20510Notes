\documentclass[../notes.tex]{subfiles}

\pagestyle{main}
\renewcommand{\chaptermark}[1]{\markboth{\chaptername\ \thechapter\ (#1)}{}}
\setcounter{chapter}{3}

\begin{document}




\chapter{Differential Forms}
\section{Overview of Differential Forms}
\begin{itemize}
    \item \marginnote{4/18:}Office Hours on Wednesday, 4:00-5:00 PM.
    \item Plan:
    \begin{itemize}
        \item An impressionistic overview of what (differential) forms do/are.
        \item Tangent spaces.
        \item Vector fields/integral curves.
        \item 1-forms; a warm-up to $k$-forms.
    \end{itemize}
    \item Impressionistic overview of the rest of \textcite{bib:DifferentialForms}.
    \begin{itemize}
        \item An open subset $U\subset\R^n$; $n=2$ and $n=3$ are nice.
        \item Sometimes, we'll have some functions $F:U\to V$; this is where pullbacks come into play.
        \item At every point $p\in U$, we'll define a vector space (the tangent space $T_p\R^n$). Associated to that vector space you get our whole slew of associated spaces (the dual space $T_p^*\R^n$, and all of the higher exterior powers $\lam[k]{T_p^*\R^n}$).
        \item We let $\omega\in\ome[k]{U}$ be a $k$-form in the space of $k$-forms.
        \item $\omega$ assigns (smoothly) to every point $p\in U$ an element of $\lam[k]{T_p^*\R^n}$.
        \item Question: What really is a $k$-form?
        \begin{itemize}
            \item Answer: Something that can be integrated on $k$-dimensional subsets.
            \item If $k=1$, i.e., $\omega\in\ome[1]{U}$, then $U$ can be integrated over curves.
        \end{itemize}
        \item If we take $k=0$, then $\ome[0]{U}=C^\infty(U)$, i.e., the set of all smooth functions $f:U\to\R$.
        \begin{itemize}
            \item \textcite{bib:DifferentialForms} doesn't, but Klug will and we should distinguish between functions $F:U\to V$ and $f:U\to\R$.
        \end{itemize}
        \item We will soon construct a map $d:\ome[0]{U}\to\ome[1]{U}$ (the \textbf{exterior derivative}) that is rather like the gradient but not quite.
        \begin{itemize}
            \item $d$ is linear.
            \item Maps from vector spaces are heretofore assumed to be linear unless stated otherwise.
        \end{itemize}
        \item The 1-forms in $\im(d)$ are special: $\int_\gamma\dd{f}=f(\gamma(b))-f(\gamma(a))$ only depends on the endpoints of $\gamma:[a,b]\to U$! The integral is \emph{path-independent}.
        \item A generalization of this fact is that instead of integrating along the surface $M$, we can integrate along the boundary curve:
        \begin{equation*}
            \int_M\dd{\omega} = \int_{\partial M}\omega
        \end{equation*}
        This is \textbf{Stokes' theorem}.
        \begin{itemize}
            \item $M$ is a $k$-dimensional subset of $U\subset\R^n$.
        \end{itemize}
        \item Note that we have all manner of functions $d$ that we could differentiate between (because they are functions) but nobody does.
        \begin{equation*}
            0 \rightarrow \ome[0]{U}
            \xrightarrow{d} \ome[1]{U}
            \xrightarrow{d} \ome[2]{U}
            \xrightarrow{d} \cdots
            \xrightarrow{d} \ome[n]{U}
            \xrightarrow{d} 0
        \end{equation*}
        \item Theorem: $d^2=d\circ d=0$.
        \begin{itemize}
            \item Corollary: $\im(d^{n-1})\subset\ker(d^n)$.
        \end{itemize}
        \item We'll define $H^k_{dR}(U)=\ker(d)/\im(d)$.
        \begin{itemize}
            \item These will be finite dimensional, even though all the individual vector spaces will be infinite dimensional.
            \item These will tell us about the shape of $U$; basically, if all of these equal zero, $U$ is simply connected. If some are nonzero, $U$ has some holes.
        \end{itemize}
        \item For small values of $n$ and $k$, this $d$ will have some nice geometric interpretations (div, grad, curl, n'at).
        \item We'll have additional operations on forms such as the wedge product.
    \end{itemize}
    \item \textbf{Tangent space} (of $p$): The following set. \emph{Denoted by} $\bm{T_p\pmb{\R}^n}$. \emph{Given by}
    \begin{equation*}
        T_p\R^n = \{(p,v):v\in\R^n\}
    \end{equation*}
    \begin{itemize}
        \item This is naturally a vector space with addition and scalar multiplication defined as follows.
        \begin{align*}
            (p,v_1)+(p,v_2) &= (p,v_1+v_2)&
            \lambda(p,v) &= (p,\lambda v)
        \end{align*}
        \item The point is that
        \begin{equation*}
            T_p\R^n \neq T_q\R^n
        \end{equation*}
        for $p\neq q$ even though the spaces are isomorphic.
        \item Aside: $F:U\to V$ differentiable and $p\in U$ induce a map $\dd{F_p}:T_p\R^n\to T_{F(p)}\R^m$ called the "derivative at $p$."
        \begin{itemize}
            \item We will see that the matrix of this map is the Jacobian.
        \end{itemize}
        \item Chain rule: If $U\xrightarrow{F}V\xrightarrow{G}W$, then
        \begin{equation*}
            \dd{(G\circ F)_p} = \dd{G}_{F(p)}\circ\dd{F_p}
        \end{equation*}
    \end{itemize}
    \item This is round 1 of our discussion on tangent spaces.
    \item Round 2, later on, will be submanifolds such as $T_pM$: The tangent space to a point $p$ of a manifold $M$.
    \item \textbf{Vector field} (on $U$): A function that assigns to each $p\in U$ an element of $T_p\R^n$.
    \begin{itemize}
        \item A constant vector field would be $p\mapsto(p,v)$, visualized as a field of vectors at every $p$ all pointing the same direction. For example, we could take $v=(1,1)$.
        \emph{picture}
        \item Special case: $v=e_1,e_2,\dots,e_n$. Here we use the notation $e_i=\dv*{x_i}$.
        \item Example: $n=2$, $U=\R^2\setminus\{(0,0)\}$. We could take a vector field that spins us around in circles.
        \item Notice that for all $p$, $\dv*{x_1}|_p,\dots,\dv*{x_n}|_p\in T_p\R^n$ are a basis.
        \begin{itemize}
            \item Thus, any vector field $v$ on $U$ can be written uniquely as
            \begin{equation*}
                v = f_1\dv{x_1}+\cdots+f_n\dv{x_n}
            \end{equation*}
            where the $f_1,\dots,f_n$ are functions $f_i:U\to\R$.
        \end{itemize}
    \end{itemize}
\end{itemize}



\section{The Lie Derivative and 1-Forms}
\begin{itemize}
    \item \marginnote{4/20:}Plan:
    \begin{itemize}
        \item Vector fields and their integral curves.
        \item Lie derivatives.
        \item 1-forms and $k$-forms.
        \item $\ome[0]{U}\xrightarrow{d}\ome[1]{U}$.
    \end{itemize}
    \item Notation.
    \begin{itemize}
        \item $U\subset\R^n$.
        \item $v$ denotes a vector field on $U$.
        \begin{itemize}
            \item Note that the set of all vector fields on $U$ constitute the vector space ??.
        \end{itemize}
        \item $v_p\in T_p\R^n$.
        \item $\omega_p\in\lam[k]{T_p^*\R^n}$.
        \item $\dv*{x_i}|_p=(p,e_i)\in T_p\R^n$.
    \end{itemize}
    \item Recall that any vector field $v$ on $U$ can be written uniquely as
    \begin{equation*}
        v = g_1\dv{x_1}+\cdots+g_n\dv{x_n}
    \end{equation*}
    where the $g_i:U\to\R$.
    \item \textbf{Smooth} (vector field): A vector field $v$ for which all $g_i$ are smooth.
    \item From now on, we assume unless stated otherwise that all vector fields are smooth.
    \item \textbf{Lie derivative} (of $f$ wrt. $v$): The function $L_vf:U\to\R$ defined by $p\mapsto D_{v_p}(f)(p)$, where $v$ is a vector field on $U$ and $f:U\to\R$ (always smooth).
    \begin{itemize}
        \item Recall that $D_{v_p}(f)(p)$ denotes the directional derivative of $f$ in the direction $v_p$ at $p$.
        \item As some examples, we have
        \begin{align*}
            L_{\dv*{x_i}}f &= \dv{f}{x_i}&
            L_{(g_1\dv{x_1}+\cdots+g_n\dv{x_n})}f &= g_1\dv{f}{x_1}+\cdots+g_n\dv{f}{x_n}
        \end{align*}
    \end{itemize}
    \item Property.
    \begin{enumerate}
        \item Product rule: $L_v(f_1f_2)=(L_vf_1)f_2+f_1(L_vf_2)$.
    \end{enumerate}
    \item Later: Geometric meaning to the expression $L_vf=0$.
    \begin{itemize}
        \item Satisfied iff $f$ is constant on the integral curves of $v$. As if $f$ "flows along" the vector field.
    \end{itemize}
    \item We define $T_p^*\R^n=(T_p\R^n)^*$.
    \item 1-forms:
    \begin{itemize}
        \item A (differential) 1-form on $U\subset\R^n$ is a function $\omega:p\mapsto\omega_p\in T_p^*\R^n$.
        \item A "co-vector field"
    \end{itemize}
    \item Notation: $\dd{x_i}$ is the 1-form that at $p$ is $(p,e_i^*)\in T_p^*\R^n$.
    \item For example, if $U=\R^2$ and $\omega=\dd{x_1}$, then we have the vector field of "unit vectors pointing to the right at each point."
    \item Note: Given any 1-form $\omega$ on $U$, we can write $\omega$ uniquely as
    \begin{equation*}
        \omega = g_1\dd{x_1}+\cdots+g_n\dd{x_n}
    \end{equation*}
    for some set of smooth $g_i:U\to\R$.
    \item Notation:
    \begin{itemize}
        \item $\ome[1]{U}$ is the set of all smooth 1-forms.
        \item Notice that $\ome[1]{U}$ is a vector space.
    \end{itemize}
    \item Given $\omega\in\ome[1]{U}$ and a vector field $v$ on $U$, we can define $\omega(v):U\to\R$ by $p\mapsto\omega_p(v_p)$.
    \item If $U=\R^2$, we have that
    \begin{align*}
        \dd{x}\left( \dv{x} \right) &= 1&
        \dd{x}\left( \dv{y} \right) &= 0
    \end{align*}
    \item Note that $\dd{x},\dd{y}$ are not a basis for $\ome[1]{U}$ since the latter is infinite dimensional.
    \item Exterior derivative for 0/1 forms.
    \begin{itemize}
        \item Let $d:\ome[0]{U}\to\ome[1]{U}$ take $f:U\to\R$ to $\pdv{f}{x_1}\dd{x_1}+\cdots+\pdv{f}{x_n}\dd{x_n}$.
        \item This represents the gradient as a 1-form.
    \end{itemize}
    \item Check:
    \begin{enumerate}
        \item Linear.
        \item $\dd{x_i}=\dd{(x_i)}$, where $x_i:\R^n\to\R$ is the $i^\text{th}$ coordinate function.
    \end{enumerate}
\end{itemize}



\section{Integral Curves}
\begin{itemize}
    \item \marginnote{4/22:}Plan:
    \begin{itemize}
        \item Clear up a bit of notational confusion.
        \item Discuss integral curves of vectors fields.
        \item $k$-forms.
        \item Exterior derivatives $d:\ome[k]{U}\to\ome[k+1]{U}$ (definition and properties).
    \end{itemize}
    \item Notation:
    \begin{itemize}
        \item $F:\R^n\to\R^m$ smooth.
        \item We are used to denoting derivatives by big $D$: $DF_p:T_p\R^n\to T_{f(p)}\R^m$ where bases of the two spaces are $e_1,\dots,e_n$ and $e_1,\dots,e_m$ has matrix equal to the Jacobian:
        \begin{equation*}
            [DF_p] =
            \begin{bmatrix}
                {\dv{F_i}{x_j}}(p)
            \end{bmatrix}
        \end{equation*}
        \item The book often uses small $d$: $f:U\to\R$ has $df_p:T_p\R^n\to T_{f(p)}\R$, where the latter set is isomorphic to $\R$.
        \item $df:p\to df_p\in T_p^*\R^n$.
        \item Klug said
        \begin{equation*}
            \dd{f} = \sum_{i=1}^n\pdv{f}{x_i}\dd{x_i}
        \end{equation*}
        \item Homework 1 defined $\dd{f}=df$?
        \item Sometimes three perspectives help you keep this all straight:
        \begin{enumerate}
            \item Abstract nonsense: The definition of the derivative.
            \item How do I compute it: Apply the formula.
            \item What is it: E.g., magnitude of the directional derivative in the direction of steepest ascent.
        \end{enumerate}
    \end{itemize}
    \item For the homework,
    \begin{itemize}
        \item Let $\omega$ be a 1-form in $\ome[1]{U}$.
        \item Let $\gamma:[a,b]\to U$ be a curve in $U$.
        \item Then $\dd{\gamma_p}=\gamma_p':T_p\R\to T_{\gamma(p)}\R^n$ is a function that takes in points of the curve and spits out tangent vectors.
        \item Integrating swallows 1-forms and spits out numbers.
        \begin{equation*}
            \int_\gamma\omega = \int_a^b\omega(\gamma'(t))\dd{t}
        \end{equation*}
        \item Problem: If $\omega=\dd{f}$, then
        \begin{equation*}
            \int_\gamma\omega = f(\gamma(b))-f(\gamma(a))
        \end{equation*}
        regardless of the path.
        \item Question: Given a 1-form $\omega$, is $\omega=\dd{f}$ for some $f$?
        \item Homework: Explicit $U$, $\omega$, closed $\gamma$ such that $\int_\gamma\omega\neq 0$ implies that $\omega\neq\dd{f}$. This motivates and leads into the de Rham cohomology.
    \end{itemize}
    \item Aside: It won't hurt (for now) to think of 1-forms as vector fields.
    \item Integral curves: Let $U\subset\R^n$, $v$ be a (smooth) vector field on $U$. A curve $\gamma:(a,b)\to U$ is an \textbf{integral curve} for $v$ if $\gamma'(t)=v_{\gamma(t)}$.
    \item Examples:
    \begin{itemize}
        \item If $U=\R^2$ and $\gamma=\dv*{x}$, then the integral curve is the line from left to right traveling at unit speed. The curve has to always have as it's tangent vector the unit vector pointing right (which is the vector at every point in the vector field).
        \item Vector fields flow everything around. An integral curve is the trajectory of a particle subjected to the vector field as a force field.
    \end{itemize}
    \item Main points:
    \begin{enumerate}
        \item These integral curves always exist (locally) and often exist globally (cases in which they do are called \textbf{complete vector fields}).
        \item They are unique given a starting point $p\in U$.
    \end{enumerate}
    \item An incomplete vector field is one such as the "all roads lead to Rome" vector field where everything always points inward. This is because integral curves cannot be defined for all "time" (real numbers, positive and negative).
    \item The proofs are in the book; they require an existence/uniqueness result for ODEs and the implicit function theorem.
    \item Aside: $f:U\to\R$, $v$ a vector field, implies that $L_vf=0$ means that $f$ is constant along all the integral curves of $v$. This also means that $f$ is integral for $v$.
    \item \textbf{Pullback} (of 1-forms): If $F:U\to V$, $d:\ome[0]{U}\to\ome[1]{U}$, and $d:\ome[0]{V}\to\ome[1]{V}$, then we get an induced map $F^*:\ome[0]{V}\to\ome[0]{U}$. If $f:V\to\R$, then $f\circ F$ is involved.
    \begin{itemize}
        \item We're basically saying that if we have $\Hom(A,X)$ (the set of all functions from $A$ to $X$) and $\Hom(B,X)$, then if we have $F:A\to B$, we get an induced map $F^*:\Hom(B,X)\to\Hom(A,X)$ that is precomposed with $F$.
    \end{itemize}
\end{itemize}




\end{document}